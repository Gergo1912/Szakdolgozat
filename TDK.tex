\documentclass[12pt]{report}

\usepackage[magyar]{babel}
\usepackage{pdfpages}
\usepackage{cite}
\usepackage{graphicx}
\graphicspath{ {./Abrak/} }
\usepackage{float}
\usepackage{listings}
\usepackage{xcolor}
\usepackage{caption}
\usepackage{amsmath}
\usepackage{geometry}
\usepackage{enumitem}
\usepackage{fancyhdr}
\usepackage{algorithm}
\usepackage{algpseudocode}
\usepackage{setspace}
\usepackage{inconsolata}
\usepackage[hidelinks]{hyperref}
\usepackage{rotating}
\usepackage{makecell}
\usepackage{amsfonts}
\usepackage{multirow}

\geometry{
 left=30mm,
 top=25.4mm,
 right=25.4mm,
 bottom=25.4mm
}
\linespread{1.5}
\pagestyle{fancy}
\fancyhf{}
\fancyhead[LO]{\footnotesize S-gráf alapú ütemező algoritmus párhuzamos hozzárendelést megengedő feladatokhoz}
\fancyhead[RO]{\footnotesize Molnár Gergő} 
\renewcommand{\headrulewidth}{0.5pt}
\fancyfoot[C]{\thepage}

\renewcommand{\listalgorithmname}{Algoritmusok jegyzéke}
\renewcommand{\lstlistingname}{Kódrészlet}
\makeatletter
\renewcommand{\ALG@name}{Algoritmus}
\makeatother
\let\Algorithm\algorithm
\renewcommand\algorithm[1][]{\Algorithm[#1]\setstretch{1.2}}

\lstset{
	basicstyle=\fontsize{9}{11}\ttfamily,	
    frame=tb,
    language=C++,
    tabsize=2	,
    numbers=left,,
    belowcaptionskip=10pt,
    showstringspaces=false,   
    breakatwhitespace=true, 
    breaklines=true,
    keepspaces=true,
    commentstyle=\color{green},
    keywordstyle=\color{blue},
    stringstyle=\color{red}
}

\begin{document}

\includepdf[pages = {1}]{kezdolap.pdf}
\pagenumbering{gobble}
\tableofcontents
\listoffigures 
\listoftables
\begingroup
\let\clearpage\relax
\listofalgorithms
\endgroup
\clearpage
\pagenumbering{arabic}
\chapter{Bevezetés}
Az ütemezés feladatával az élet számos területén találkozunk, kezdve az egyszerű, hétköznapi problémáktól, mint például egy napon elvégzendő feladataink sorrendjének beosztása, professzionális sportcsapatok heti edzésprogramjának kialakításán át, egészen az ipari üzemek működéséig, ahol a rendelkezésre álló berendezésekhez kell rendelni az előállítani kívánt termékeket.
Bár az életünk különböző területein fellépő ütemezési problémák különböznek egymástól, bizonyos mértékben hasonlóság is fellelhető közöttük.
Eltérés lehet az ütemezési feladatok célfüggvénye.
Továbbá az adott ütemezési probléma lehet online, offline, illetve sztochasztikus vagy determinisztikus.
Minden fellépő probléma esetében az a cél, hogy az elvégzendő feladatokat a rendelkezésre álló erőforrások között megosszuk oly módon, hogy adott intervallumon belül a lehető legjobb megoldást kapjuk.
Fontos az, hogy ezt úgy tegyük meg, hogy a folyamat során fellépő korlátokat betartjuk, azokat nem sértjük meg.
Az ipari ütemezés két leggyakoribb célja a makespan minimalizálás, és a throughput maximalizálás.
A szakirodalomban legtöbb esetben a \textit{makespan} kifejezést az idő minimalizálására, a \textit{throughput} kifejezést pedig a termelés során előállított mennyiség maximalizálására alkalmazzák.

Az ipari gyártási folyamatok ütemezésére különböző módszerek léteznek már.
Ezek közé tartoznak a MILP (Mixed Integer Linear Programming - vegyes egészértékű lineáris programozás) megoldó módszerek, amelyek a lineáris programozáson alapulnak.
Továbbá ide sorolhatóak az időzített automaták, Petri hálók, valamint az S-gráf megoldó módszer, amely a munkám során a legfontosabb szerepet tölti be a felsorolt megoldó módszerek közül.
Dolgozatom második fejezetében ezek a módszerek kerülnek bemutatásra.
A harmadik részben a probléma definiálása történik meg, a negyedikben pedig az általam megvalósított módszer elmélete található.
Az ötödik fejezetben a módszer megvalósítását és szoftverbe való beillesztését mutatom be.
A hatodik fejezetben tesztelésről, és az eredmények összehasonlításáról lesz szó.
A dolgozat végén található a munkám összefoglalása, a hivatkozások, valamint a függelék.
\chapter{Irodalmi áttekintés}
\section{Ipari ütemezési feladatok}
Az ipari folyamatokat gyártásnak nevezzük, amely során az elkészítendő termék létrehozása, megvalósítása a feladat. Ehhez szükség van arra, hogy megfelelően vegyük igénybe a rendelkezésre álló erőforrásokat, amelyeket berendezéseknek, unitnak nevezünk a gyártási feladtok során. A folyamat során fellépő feladatokra a taszk elnevezés is használható. Minden ütemezési feladat rendelkezik végrehajtási idővel, ami megmutatja mekkora időtartam alatt valósítható meg. Ezenfelül lehet még a feladatoknak olyan időkorlátja, ami alatt kötelező elvégezni a feladatot, ezt időhorizontnak, time horizontnak hívunk. A problémák kimenetelük szerint lehetnek megvalósíthatatlan (infeasible) és megvalósítható (feasible) feladatok. Ha egy korlátozásnak sem felel meg az adott probléma akkor infeasible, egyéb esetekben pedig már megvalósítható lesz.

Az ipari folyamatokat többféleképpen lehet csoportosítani. Egy fajta az, amelyben folyamatos és szakaszos üzemű rendszerek csoportjára bontjuk őket. Az első típusban az anyag folyamatosan kerül a rendszerbe, a másodikban pedig ez a folyamat lépésekben valósul meg. A munkám az utóbbi típusba tartozó feladatokra koncentrál. Másik lehetséges felosztás az, amikor online, offline, és semi-offline kategóriákba vannak a feladatok besorolva. Az offline esetben minden szükséges bemeneti adat rendelkezésre áll az optimalizálás idejében. Az online ezzel szemben úgy működik, hogy előbb kell döntéseket meghozni, minthogy adott paraméterekhez tartozó értékekre fény derülne. A semi-offline a kettő közé sorolható. Bizonyos információk, adatok már rendelkezésre állnak, mások viszont nem. Egy harmadik besorolási lehetőség, hogy megkülönböztetünk sztochaikus és determinisztikus feladatokat. Sztochaikus esetekben a paraméterek már futás közben kapnak értéket. Ezzel szemben a determinisztikus feladatok értékei előre meg vannak határozva és beállítva.

Az ütemezési feladatok modelljét receptnek nevezzük. Egy termék receptje tartalmazza az adott recept által előállítható termék elkészítéséhez szükséges információkat\cite{Hegyhati}. Egy receptet következő elemek együtt alkotják meg:
\begin{itemize}
  \item Termékek listája
  \item Taszkok listája, amelyek adott sorrendben történő elvégzése szükséges a termékek előállításához
  \item Taszkok egymás közötti sorrendisége, amely megmutatja a taszkok sorrendjét
  \item Rendelkezésre álló berendezések
  \item A lehetséges taszk-berendezés párok feldolgozási ideje
\end{itemize}
A recepteket a feladatok precedenciájuk szerint következő csoportokba lehet besorolni. A felsorolás a legegyszerűbbtől halad az általánosabb felé. Ezen kívül minden osztály a következőnek egy speciális esete.
\begin{itemize}
	\item \textbf{Single Stage:} Egy lépésben állítható elő minden egyes termék.
	\item \textbf{Simple Multiproduct:} Minden terméket meghatározott számú fázison, szakaszon keresztül lehet elkészíteni. Előzővel szemben itt már nem csak egy lépésben lehetséges.
	\item \textbf{General Multiproduct:} Előzővel összehasonlítva a különbség, hogy ennél lehetséges lépések kihagyása.
	\item \textbf{Multipurpose:} A termékek gyártásának lépéseit nem lehet egy balról jobbra tartó vonal mentén véghezvinni. A szakaszok száma és iránya tetszőleges, sőt egy szakasz többször is ismétlődhet ugyanabban a gyártásban.
	\item \textbf{Precedential:} A gyártásban résztvevő taszkok nincsenek hozzárendelve a szakaszokhoz. Egy termék gyártás nem szükségszerűen lineáris, lehetnek elágazások, kör azonban nem megengedett. Minden taszk előfeltételét be kell fejezni mielőtt az adott lépés megkezdődik. 
	\item \textbf{General Network:} A legáltalánosabb recept, ahol a taszkok a bemenetük és a kimenetük által adottak. Ennél az esetén kör is lehetséges.
\end{itemize}
Néhány előbb említett feladattípus szemléltetése látható a~\ref{receptek} ábrán. A receptek jobb oldalán lévő kör jelenti a terméket, a többi pedig a taszkokat.
\begin{figure}[H]	
\begin{center}
\includegraphics[scale=0.7]{receptek}
\caption{Különböző receptek szemléltetése}
\label{receptek}
\end{center}
\end{figure}
\newpage
A vegyipari, gyártási ütemezési feladatoknál nagy szerepet játszik a tárolási irányelv. A tárolási irányelvek azt mutatják meg, hogy két egymást követő feladatok között az elkészített köztes termékeket, hogyan kell raktározni, tárolni, illetve ez mennyi ideig lehetséges. A tárolási stratégiák csoportosításra többféle lehetőség van. Egyik ezek közül, amikor az adott létesítmény infrastrukturális képességei korlátozzák az anyag mennyiségét és minőségét.
\begin{itemize}
	\item \textbf{UIS - Unlimited Intermediate Storage}
	\item \textbf{FIS - Finite Intermediate Storage}
	\item \textbf{NIS - No Intermediate Storage}
\end{itemize}
Az UIS eset a legegyszerűbb. Ebben az esetben van lehetőség a köztes anyagok bármely mértékű tárolására. FIS esetben van lehetőség a tárolása, de csak korlátozott mennyiségben. A NIS esetében nincs külön tárolásra alkalmas egység, de az megoldható, hogy amíg a következő feldolgozó egységhez kerül, addig az előző helyen várakozzon.

Második fajta csoportosítási lehetősége, amikor az idő ad korlátot.
\begin{itemize}
	\item \textbf{UW - Unlimited Wait}
	\item \textbf{LW - Limited Wait}
	\item \textbf{ZW - Zero Wait}	
\end{itemize}
ZW esetben nincs lehetőség a köztes anyag tárolására, azaz ha a berendezés befejezte a munkát, akkor azonnal folytatni kell a gyártást. AZ LW esetben van egy idő, amíg a köztes termék várakozhat. Azonban, ha ez a rendelkezésre álló idő elfogy, akkor muszáj folytatni a gyártás folyamatát. Az UW eset a legegyszerűbb mind közül. Ha a köztes anyag tulajdonságai lehetőséget biztosítanak, akkor a tárolási idő nincs korlátozva, bármennyi ideig lehetőség van a tárolásra, raktározásra.
\newpage
\section{Megoldó módszerek}
Az ütemezési feladatok megoldására számos megoldó módszer létezik. Ezek közül a legismertebbek, és legszélesebb körben elterjedt módszerek kerülnek bemutatásra a dolgozatom következő pontjaiban.
\subsection{MILP modellek}
A legtöbb ütemezési probléma modelljében vegyesen fordulnak elő folytonos és egész változók. Ilyen esetekben beszélünk vegyes egészértékű lineáris programozásról, azaz \textbf{M}ixed \textbf{I}nteger \textbf{L}inear \textbf{P}rogramming. Több altípus létezik:
\begin{itemize}
  \item[] \textbf{Időfelosztásos modellek - Time discretization based:} A módszer időpontokat és időréseket határoz meg. Ezek jelentek meg legkorábban kronológiailag \cite{kondili}. Az időrésen és az időponton alapuló megközelítések \cite{susarla} sok hasonlóságot mutatnak, mivel egy időintervallumtól egy másikig terjedő időintervallumot tekinthetünk időrésnek. Ellenkező irányból nézve pedig egy időrés kezdő időpontját tekinthetjük egy időpontnak.  
  
  Minden időpontban bináris változók vannak hozzárendelve aszerint, hogy az adott időpillanatban kezdődik a feladat végrehajtása vagy nem. A bináris változók száma arányos lesz az időpontok számával. A szándék mindig megvolt olyan módszer kifejlesztésére, amelyben a szükséges időpontok száma minél kisebb legyen amellett, hogy megtalálja az optimális megoldást. 
  
  \item[] \textbf{Precedencia alapú modellek - Precedence based:} Ezeknél a módszereknél, szemben az időfelosztásos módszerekkel, nincs szükség az időhorizont diszkretizációjára, nem használnak ismeretlen paramétert a modellben. Általánosságban jobb számítási eredményeket nyújtanak az általuk kezelt problémákra, azonban ez a készlet sokkal kisebb, mint az időfelosztásos modellekhez tartozó kollekció. Alapvetően a multiproduct és multipurpose receptek esetében használható megfelelően, de kibővíthető, hogy a sokkal általánosabb precedential receptek is megoldhatók legyenek. 
  
  Ez a módszer kettő darab bináris változó használ. Az első $Y_{i,j}$, aminek az értéke abban az esetben lesz 1, ha $i$ feladatot $j$ berendezés végzi el. A második változó: $X_{i,j,i'}$. Értéke akkor lesz 1, ha ugyan az a berendezés végzi el az $i$ és $i'$ taszkot, méghozzá úgy, hogy előbb az $i$-t teljesíti. 
\end{itemize}
\subsection{Analízis alapú eszközök}
Az automatákat és Petri hálókat széles körben alkalmazzák diszkrét eseményrendszerek modellezésére \cite{cassandras}. Számos kísérletet tettek ezen eszközök modellezési teljesítményének kiterjesztésére annak érdekében, hogy batch folyamatok ütemezésére is alkalmassá tegyék. A meglévő modelleket időzítéssel egészítették ki, így jöttek létre, a Timed Place Petri Nets (TPPN) and Timed Priced Automata (TPA) módszerek, amelyek Branch and Bound algoritmus használnak azért, hogy a legelőnyösebb megoldást megtalálják. Ezen módszereknek a hatékonysága elmarad a MILP és az S-gráf modell hatékonyságától is.
\begin{itemize}
	\item[] \textbf{Időzített automaták:} Ezekben a megközelítésekben a recepteket és a berendezéseket külön modellezik, és a rendszer modellje ezeknek a párhuzamos összetételével jön létre. A bonyolultságos az jelenti, hogy az órák állapota végtelen lehet, és emiatt a rendszer állapotterülete is az lehet.
	\item[] \textbf{Időzített Petri háló:} Az alap ilyen módszereknél, hogy az átvitel jele késleltetés alapján jön létre. Többen is foglalkoztak a témával, például Ghaeli \cite{ghaeli}, aki batch folyamatok ütemezésével foglalkozott.
\end{itemize}

\subsection{S-gráf módszertan}
Az S-gráf keretrendszer volt az első olyan módszer, amely publikált gráf elméleten alapult, valamint szakaszos gyártórendszerek ütemezési problémák megoldására szolgált.\cite{combtech} Ez a keretrendszer egy irányított gráf modellből, az S-gráfból, és a hozzá tartozó algoritmusokból áll. \cite{combframe} Az S-gráf egy speciális irányított gráf, ami ütemezési problémák számára. Nemcsak a recept vizualizációja, hanem egyben matematikai modell is. A keretrendszerben az S-gráf reprezentálja a recepteket, a részleges és a teljes ütemterveket is. Ezekben a gráfokban a termékeket és a feladatokat csúcsok jelölik, amelyeket csomópontoknak (node) nevezünk. Ha két feladat között összeköttetés van, akkor ezt a gráfon lévő, feladatokat reprezentáló csomópontok között lévő nyíl mutatja. Ütemezési döntés nélküli S-gráfot \textbf{Recept gráfnak} nevezzük. Erre példa a~\ref{receptGraf} ábrán\cite{Hegyhati} látható. 

A jobb oldalon látható három, nagybetűvel jelölt csomópont felel meg a termékeknek, a maradék kilenc pedig a részfeladatokat jelenti. Ezt a kilenc részfeladatot el kell végezni a termékek előállításának érdekében. Az élek (nyilak) a csomópontok közti függőséget mutatják meg. Ezeket \textbf{Recept éleknek} nevezzük. Kétfajta függőséget tudunk megkülönböztetni:
\begin{itemize}
  \item Két részfeladat között van él. Ebben az esetben az egyik készíti el a másiknak a bemenetét.
  \item Egy termék és egy részfeladat között szerepel él. Ilyenkor a részfeladat készíti el a terméket.
\end{itemize}
A nyilakon látható súlyok a részfeladat végrehajtásához szükséges időt mutatják meg. Ha egy részfeladatot több berendezés is képes elvégezni, akkor az előbb említett súly mindig a legkisebb előállítási idő lesz.
\begin{figure}[H]	
\begin{center}
\includegraphics[scale=0.7]{receptGraf}
\caption{A recept gráf szemléltetése}
\label{receptGraf}
\end{center}
\end{figure}
Minden S-gráf-hoz kapcsolódó algoritmus kiegészíti ezeket a gráfokat az úgynevezett \textbf{ütemezési élekkel}, amelyek az ütemezési döntést testesítik meg. Ezekkel az élekkel kiegészített gráfoknak a neve \textbf{Ütemezési gráf}. Példa a~\ref{utemezesiGraf} ábrán nézhető meg. Az ábrán sötétkékkel jelölt nyilak az ütemezési élek. Az ütemezési élek súlya alapértelmezetten nulla, ha a probléma nem tartalmaz szállítási, vagy tisztítási időt. A részfeladatok csomópontjain már nem a lehetséges berendezések halmaza látható, hanem egy konkrét kiválasztott berendezés, az ütemezési döntésnek megfelelően. 
\begin{figure}[H]
\begin{center}
\includegraphics[scale=0.7]{utemezesiGraf}
\caption{Az ütemezési gráf szemléltetése}
\label{utemezesiGraf}
\end{center}
\end{figure}
Ugyanahhoz a berendezéshez rendelt részfeladatok végrehajtási sorrendje könnyedén leolvasható a gráfról. A~\ref{utemezesiGraf2} ábrán látható példában az E2-es berendezés által elvégzett részfeladatok sorrendje B1 -\textgreater  C2 -\textgreater  A3. Ahhoz, hogy a berendezés el tudjon végezni egy adott részfeladatot nem elég, hogy az általa végzett előző részfeladatot befejezze, hanem szükség van még ezen kívül arra, hogy a soron következő részfeladathoz szükséges összes részfeladat elkészüljön. Csak ezeket követően tudja megkezdeni az adott részfeladat végrehajtását. Példa: C2-es részfeladat végrehajtásához szükséges, hogy az E2-es berendezés befejezze a B1-es részfeladatot, valamint az is, hogy a C1-es részfeladatot elkészítse az E1-es berendezés.
\begin{figure}[H]
\begin{center}
\includegraphics[scale=0.7]{utemezesiGraf2}
\caption{E2-es berendezés által elvégzett részfeladatok}
\label{utemezesiGraf2}
\end{center}
\end{figure}
Gyakran használt mód az ütemezési feladatok ábrázolására a Gantt diagram\cite{ganttwwf}\cite{ganttofw}. Ezeken a diagramokon a függőleges tengelyen a berendezések, míg a vízszintes tengelyen a pedig az idő szerepel. Az ábrán látható erőforrások szemléltetik az erőforrások elfoglaltságát. Egy Gantt diagram látható a~\ref{GanttDiagram} ábrán.
\begin{figure}[H]
\begin{center}
\includegraphics[scale=0.7]{GanttDiagram}
\caption{Egy ütemezés Gantt diagramon való megjelenítése}
\label{GanttDiagram}
\end{center}
\end{figure}

\subsection{A makespan minimalizálás algoritmusa}
Az algoritmus (\ref{makespanKod} ábra) első lépésben inicializálja a $makespan^{cb}$ értékét végtelennel, majd beállítja a $S$ halmazt, amelyben az ütemezés során a nyitott részproblémák szerepelnek. Kezdetben csak a gyökér probléma szerepel benne, vagyis egy recept gráf bármilyen hozzárendelés nélkül. A \textbf{recipe} függvény visszaadja a $G(N,A_1,A_2,w)$ által jelzett probléma recept gráfját, ahol


Az iteráció minden lépésében a \textbf{select\_remove} függvénnyel egy tetszőleges részprobléma kerül kiválasztásra, majd az $S$-ből eltávolításra. Ennek a függvénynek a viselkedése a különböző megvalósításokban más és más lehet, ami más keresési stratégiát eredményez.

Az iteráció elején kiértékelődik, hogy a részprobléma képes-e optimális megoldást nyújtani, vagy sem. Ez a \textbf{bound} függvénnyel történik. A leggyakrabban a leghosszabb út keresésével vizsgálja meg a részproblémát \textbf{bound} függvény, de lehetséges LP alapú modellek használata is. Ha a korlát nem kisebb, mint az eddig megtalált legjobb eredmény, akkor az iteráció véget ér, és a következő részprobléma kerül kiválasztásra, amennyiben létezik.

Ha viszont kisebb a korlát, akkor ellenőrzi az algoritmus azt, hogy az összes taszk már ütemezett-e, azaz a részprobléma teljesen ütemezett. Ha így van a legjobb megoldás frissítésre kerül. Ha még szükséges további ütemezés, akkor a  \textbf{select} függvény kiválaszt egy rendelkezésre álló berendezést a $J'$ halmazból. A kiválasztott $j$ berendezéshez az algoritmus hozzárendeli az összes lehetséges taszkot. Ezek kapnak egy másolatot az aktuális S-gráfról. Ezt kibővíti az új hozzárendelés alapján az ütemezési élekkel. Végezetül pedig az új részproblémát hozzáadja az $S$ halmazhoz. Ha ez az $S$ halmaz üres lesz, akkor a $G^{cb}$ és a hozzárendelések leírják az optimális megoldást, és ha legalább egy megvalósítható akkor az algoritmus visszatér ezzel az értékkel. Ellenkező esetben nem ad vissza semmit.

\begin{figure}[H]
\begin{center}
\includegraphics[scale=1]{makespanKod}
\caption{A makespan minimalizáló algoritmus pszeudó kódja\cite{Hegyhati}}
\label{makespanKod}
\end{center}
\end{figure}

\section{Throughput vagy profitmaximalizálás}
Eredetileg az S-gráf keretrendszer makespan minimalizációs problémák megoldására lett létrehozva. Azonban a későbbiekben bővítésre került, így ezután throughput, profitmaximalizációs problémák megoldására is lehet alkalmazni. Az alapötlet Majozi and Friedler \cite{majozifriedler}, valamint Holczinger \cite{holczinger} nevéhez fűződik.
\begin{figure}[H]
\begin{center}
\includegraphics[scale=1]{throughput_alg}
\caption{A jövedelem maximalizáló algoritmus pszeudó kódja \cite{Hegyhati}}
\label{throughput_alg}
\end{center}
\end{figure}
Az algoritmus először inicializálja S halmazt, minden lehetséges batch számmal, a termékekre vonatkozóan. Fontos kiemelni, hogy ebben az esetben minden terméknél a batch méret rögzített. Ezt követően minden iteráció során az előbb említett halmazból kiválasztásra kerül egy konfiguráció a \textbf{select\textunderscore remove} függvény segítségével. Ezután kerül feasibiliy tesztelésre, amely során eldől, hogy a megadott időhorizont alatt megvalósítható vagy sem. Ha megvalósítható és nagyobb jövedelmet biztosít, akkor a jelenlegi legjobb megoldás felülíródik. Ha infeasible a kiválasztott konfiguráció, akkor ez, és minden ennél nagyobb konfiguráció eltávolításra kerül az S halmazból. Amint az S halmaz üressé válik, és volt feasible megoldás, akkor az algoritmus visszatér a legjobb konfigurációval, és az ehhez tartozó jövedelem mennyiségével.

A~\ref{throughput_szemleltetes} ábrán látható egy Throughput módszerrel megvalósított feladat eredménye.
\begin{figure}[H]
\begin{center}
\includegraphics[scale=0.65]{throughput_szemleltetes}
\caption{Throughput maximalizálás szemléltetés}
\label{throughput_szemleltetes}
\end{center}
\end{figure}
Az algoritmus először végighalad a tengelyek mentén, vagyis az egyik termék batch mérete 0 lesz, a másik pedig növekszik. Ezt addig folytatja amíg megtalálja az első nem megvalósítható, azaz infeasible konfigurációt. Ezt követően megteszi ezt a másik tengelyen is. Így kap egy jelenleg maximális jövedelmet. Az utolsó még feasible batch méreteknél nagyobb batch mérteket már nem kell vizsgálni, hiszen azokat a megadott időhorizonton belül nem lehet megvalósítani. A képen látható feladatban ez 16 volt. A két tengelyen lévő 16 értékhez tartozó batch méretet összeköti egy vonallal. Jelenlegi példában a B termékhez tartozó 16-os érték nem egész batch mérethet tartozik, a legközelebbi a 15-tel rendelkező, ami infeasible már. A behúzott vonal alatt azok a konfigurációk szerepelnek, amelyek jövedelme nem éri ez a jelenlegi maximumot, így ezek nem a lehető legjobb megoldást adják meg. Ezek tesztelésére már nem kell időt fordítani. Az algoritmus addig folytatja a futást, amíg a halmaz, amely a konfigurációkat tartalmazza ki nem ürül. Ha nem talált megvalósítható konfigurációt, akkor az adott problémát a megadott időhorizontot belül nem lehet megoldani. Ellenkező esetben az algoritmus megadja az elérhető legnagyobb profitot, és a konfigurációt, amellyel ez elérhető. A példafeladatban a maximálisan elérhető jövedelem 25. Ezt négy darab A termék és 3 darab B termék legyártásával lehet elérni.
\chapter{Probléma definíció}
Számos esettanulmány és irodalmi példa esetében a batch méretek nem rögzítettek, és ha több egység képes elvégezni egy részfeladatot, akkor párhuzamosan tehetik meg azt. Ez a fajta probléma meghatározás különösen gyakran a throughput vagy bevétel maximalizációs probléma esetén jelentkezik, de megjelenhet a makespan minimalizálásnál is. Az idő diszkréción alapuló megközelítések meg tudják oldani ezt a problémát, azonban az S-gráf keretrendszer esetén néhány módosítás szükséges. Az ~\ref{throughput_alg} ábrán látható algoritmus megköveteli, hogy a recept rögzített legyen, valamint egy termékhez tartozó batch jövedelme is előre ismert legyen. Az előbb említett problémák esetén viszont egyik sem garantált. 
A javasolt megközelítés szemléltetésére a Kondili és munkatársaitól származó példát veszem igénybe \cite{kondili}. Ez látható \ref{kondiliPelda} ábrán.

A folyamat 5 taszkból áll: fűtésből, 3 darab reakcióból, és az elválasztásból. Ezekhez 4 berendezés áll rendelkezésre: a fűtőtest és szeparátor, mindkettő 100 kilogrammos kapacitással a fűtés és elválasztás folyamatához. A három reakciós folyamathoz van 2 darab reaktor ugyanakkora feldolgozási idővel. A kapacitásuk eltér, egyiké 80 kilogramm a másiké pedig 50 kilogramm, de ezeket a reaktorokat párhuzamosan is igénybe lehet venni. Feltételezzük, hogy az összes egység képes a kapacitásuknál kisebb terheléssel működni, azaz nincs a teherbírásuknak alsó határa. A folyamat során két termék készül azonos profittal. A következő korlátozások fennállnak: nem marad semmilyen köztes anyag a termelő folyamat végén, nem lehet csak az egyes számú terméket gyártani, illetve nincs tárolásra lehetőség a folyamat során.
\begin{figure}[H]
\begin{center}
\includegraphics[scale=0.65]{kondiliPelda}
\caption{Kondili példafeladata}
\label{kondiliPelda}
\end{center}
\end{figure}
Mindegyik reakciós folyamat elvégezhető egyik vagy másik reaktor által, illetve ezekkel párhuzamosan. Ez $3^3 = 27$ fix receptet eredményez, amelyek különböző batch méret intervallummal rendelkezhetnek. Mindegyik esetben különálló S-gráf receptet kell létrehozni, hogy a korábban említett S-gráf algoritmust igénybe lehessen venni profit maximalizálásra, így a keresési terület 27 dimenziós térré válna. Ez óriási CPU igényhez vezetne az optimalizálás során, ezért az esetek számának csökkentése elengedhetetlen.

\begin{table}[H]
	\begin{center}
		\caption{A 27 állandó recept Kondili példájához}
		\captionsetup[table]{skip=10pt}
		\label{tabla1}
		\begin{tabular}{r|ccc|l}
		Eset & Reakció 1 & Reakció 2 & Reakció 3 & Max bevétel  \\ 
		\hline
		1    & R1        & R1        & R1        & 86,00        \\
		2    & R1        & R1        & R2        & 71,67        \\
		3    & R1        & R1        & R1\&R2    & 86,00        \\
		4    & R1        & R2        & R1        & 53,75        \\
		5    & R1        & R2        & R2        & 53,75        \\
		6    & R1        & R2        & R1\&R2    & 53,75        \\
		7    & R1        & R1\&R2    & R1        & 114,76       \\
		8    & R1        & R1\&R2    & R2        & 71,67        \\
		9    & R1        & R1\&R2    & R1\&R2    & 139,75       \\
		10   & R2        & R1        & R1        & 86,00        \\
		11   & R2        & R1        & R2        & 71,67        \\
		12   & R2        & R1        & R1\&R2    & 86,00        \\
		13   & R2        & R2        & R1        & 53,75        \\
		14   & R2        & R2        & R2        & 53,75        \\
		15   & R2        & R2        & R1\&R2    & 53,75        \\
		16   & R2        & R1\&R2    & R1        & 89,58        \\
		17   & R2        & R1\&R2    & R2        & 71,67        \\
		18   & R2        & R1\&R2    & R1\&R2    & 89,58        \\
		19   & R1\&R2    & R1        & R1        & 86,00        \\
		20   & R1\&R2    & R1        & R2        & 71,67        \\
		21   & R1\&R2    & R1        & R1\&R2    & 86,00        \\
		22   & R1\&R2    & R2        & R1        & 53,75        \\
		23   & R1\&R2    & R2        & R2        & 53,75        \\
		24   & R1\&R2    & R2        & R1\&R2    & 53,75        \\
		25   & R1\&R2    & R1\&R2    & R1        & 114,76       \\
		26   & R1\&R2    & R1\&R2    & R2        & 71,67        \\
		27   & R1\&R2    & R1\&R2    & R1\&R2    & 139,75      
		\end{tabular}
	\end{center}
\end{table}

Ha megnézzük a \ref{tabla1} táblázatot észrevehető, hogy csupán néhány érték ismétlődik. Ennek oka az anyagok egyensúlyából származik. Például, hogy ha mind az R1-et mind az R2-t hozzárendeljük a hármas számú reakcióhoz ahelyett, hogy csak az R1 lenne hozzárendelve, akkor sem lesz nagyobb a kimenet, mert a korábbi reakciókból származó pótlás nem éri el a szükséges szintet. Ha a két különböző eset, $c$ és $c'$, ugyanakkora maximális jövedelemmel rendelkezik, de a $c$ eset csak kisebb részét használja a $c'$ által használt egységeknek, ekkor az mondjuk, hogy a $c$ \textit{dominálja} a $c'$-t. A példából látható, hogy a 9-es eset dominálja a 27-es esetet. Továbbá a 24-es eset dominálva van a 4, 5, 6, 13, 14, 15, 22, 23 esetek által. Megállapíthatjuk, hogy ha egy eset legalább egy másik által dominálva van, akkor azt kizárhatjuk a vizsgálatból, mert az továbbra is garantálva van, hogy megtalálja az optimális megoldást. A \ref{tabla3} táblázat tartalmazza azokat az eseteket, amelyek nincsenek dominálva más esetek által. 	

\begin{table}[H]
	\begin{center}
		\caption{Nem dominált esetek bevétel szerint növekvő sorrendben}
		\captionsetup[table]{skip=10pt}	
		\label{tabla2}	
		\begin{tabular}{r|ccc|l}
		Eset & Reakció 1 & Reakció 2 & Reakció 3 & Max bevétel  \\ 
		\hline
		4    & R1        & R2        & R1        & 53,75        \\
		5    & R1        & R2        & R2        & 53,75        \\
		13   & R2        & R2        & R1        & 53,75        \\
		14   & R2        & R2        & R2        & 53,75        \\
		2    & R1        & R1        & R2        & 71,67        \\
		11   & R2        & R1        & R2        & 71,67        \\
		1    & R1        & R1        & R1        & 86,00        \\
		10   & R2        & R1        & R1        & 86,00        \\
		16   & R2        & R1\&R2    & R1        & 89,58        \\
		7    & R1        & R1\&R2    & R1        & 114,67       \\
		9    & R1        & R1\&R2    & R1\&R2    & 139,75      
		\end{tabular}
	\end{center}
\end{table}

Ezután a esetszám csökkenés után is még mindig 11 esetet kellene az S-gráf algoritmusnak megvizsgálni. Annak érdekében, hogy tovább csökkenjen ez a szám több esetet is össze lehet vonni. Például a 4-es és 5-ös eset teljes mértékben megegyezik, azzal a kivétellel, hogy a harmadik reakciós folyamatot más reaktor végzi. Ezt a két esetet össze lehet vonni úgy, hogy a harmadik reakciónál R1 vagy R2-es reaktor ($R1 \vee R2$) üzemel. A \ref{tabla3} táblázatban látható a végleges összevonás eredménye.

\begin{table}[H]
	\begin{center}
		\caption{Összevont, nem dominált esetek bevétel szerint növekvő sorrendben}
		\captionsetup[table]{skip=10pt}	
		\label{tabla3}	
		\begin{tabular}{r|ccc|l}
		Eset      & Reakció 1~ & Reakció 2 & Reakció 3 & Max bevétel  \\ 
		\hline
		4,5,13,14 & $R1 \vee R2$      & R2 & $R1 \vee R2$     & 53,75        \\
		2,11      & $R1 \vee R2$      & R1        & R2        & 71,67        \\
		1,10      & $R1 \vee R2$      & R1        & R1        & 86,00        \\
		16        & R2         & R1\&R2    & R1        & 89,58        \\
		7         & R1         & R1\&R2    & R1        & 114,67       \\
		9         & R1         & R1\&R2    & R1\&R2    & 139,75      
		\end{tabular}
	\end{center}
\end{table}





\chapter{Az új módszer}
Az új algoritmus alapját a \ref{makespanKod} ábrán látható makespan minimalizáló algoritmus képezi. Az újonnan létrehozott algoritmus korlátját már nem a makespan adja, hanem a jövedelmek. További változtatás, hogy egy taszkot nem csak egy berendezés tud végrehajtani, hanem a különböző kapacitású berendezések képesek az adott taszkokat párhuzamosan is végezni. A függvényt \aref{parhuzamos}. algoritmus mutatja be.

Az algoritmus bemenetét az időhorizont ($TH$), valamint a batch méret adja. Első lépésként a $profit^{cb}$ értékét inicializálja mínusz végtelennel, mert ennél biztosan nagyobb lesz a profit megoldható probléma esetén. A $SOAA$ halmaz kezdetben nem tartalmaz értéket, később futás során azok a berendezések lesznek benne, amelyek már hozzá vannak rendelve egy adott taszkhoz. A \textbf{recipe} függvény ugyanúgy, ahogy a makespan minimalizáló algoritmus esetében van, visszaadja a $G(N,A_1,A_2,w)$ által jelzett probléma receptgráfját, ahol
\begin{itemize}
  \item[] $N$: taszkokat és termékeket reprezentáló csomópontok halmaza	
  \item[] $A_1$: a receptélek halmaza
  \item[] $A_2$: ütemezési élek halmaza, ebben a pillanatban még üres
  \item[] $w_{i,i'}$: receptélek súlya, ami az $i$ taszk minimális feldolgozási ideje.
\end{itemize} 

A \textbf{select\_remove} függvény kiválaszt egy tetszőleges részproblémát. Ezzel egyidejűleg az $S$-ből kikerül a kiválasztott részprobléma. A kiválasztott részprobléma a $PP$ változóba kerül a következő formában  $(G(N,A_1,A_2,w),I',J',{\cal A})$, ahol 
\begin{itemize}
  \item[] $(G(N,A_1,A_2,w)$: az ütemezési gráf	
  \item[] $I'$: a nem ütemezett taszkok halmaza
  \item[] $J'$: azon berendezések halmaza, amelyeket lehet még taszkokhoz rendelni
  \item[] ${\cal A}$: taszk-berendezés hozzárendelések $(i,j)$ párok formájában.
\end{itemize}

Ezek után az algoritmus megvizsgálja a kiválasztott részproblémát a \textit{Feasible} függvény segítségével. Ez megnézi, hogy a részprobléma a megadott időhorizonton belül megvalósítható-e? Ha igen, akkor megvizsgálja az algoritmus, hogy a részprobléma által nyújtott legnagyobb elérhető jövedelem nagyobb-e, mint az eddig megtalált legjobb megoldás. Ha valamelyik feltételnek nem felel meg a részprobléma, akkor az iteráció ezen lépése véget ér, és egy másik részprobléma kerül kiválasztásra, amennyiben még van ilyen.

Ha mindkét feltételnek megfelel a részprobléma, akkor megnézi az algoritmus, hogy levél-e. Ez azt jelenti, hogy található-e még olyan berendezés, amelyet lehet taszkhoz rendelni. Abban ez esetben ha nincs ilyen berendezés ($J'== \emptyset$), azaz a probléma levélnek minősül, akkor a jelenlegi legjobb megoldás ($profit^{cb}$) értéke felülíródik a vizsgált részprobléma által nyújtott megoldással. Ha viszont nem levél, akkor egy berendezés kerül kiválasztásra a még elérhető berendezések halmazából. A kiválasztott $j$ berendezéshez az algoritmus hozzárendeli az összes olyan lehetséges taszkot, ami még nem volt korábban hozzárendelve. Minden ilyen taszkhoz készül egy másolat a jelenlegi S-gráfról, ami kiegészül az új hozzárendelésekkel.

Azon taszkok esetében, amelyeket még egy berendezés sem végez, viszont a kiválasztott $j$ berendezésen kívül más berendezés is meg tud oldani, olyan döntés is születhet, hogy az adott berendezés nem is fogja azt elvégezni. Ilyenkor az $S$ halmaz kibővül egy olyan részproblémával, ahol a $j$ berendezés már nem végez több taszkot. 

Az ütemezés elvégeztével az algoritmus visszatér a megtalált legjobb megoldással, vagyis az elérhető legnagyobb bevétel értékével.

\newpage
\underline{Az algoritmus változóinak elnevezése a következő:}
\begin{itemize}
	\item $profit^{cb}$: a megtalált legjobb megoldás
	\item $SOAA$ (Set Of Already Assigned): azon taszkok halmaza, amelyek már hozzá vannak rendelve az adott berendezéshez
	\item $S$: a nyitott részproblémák halmaza
	\item $PP$: az aktuális részprobléma
	\item $I$: a taszkok halmaza
	\item $J$: a berendezések halmaza
	\item $TH$: az időhorizont
	\item $A$: az élek halmaz
	\item $w$: a receptél súlya
	\item $profit\_bound$: a kiválasztott részprobléma profit korlátja.
\end{itemize}

\underline{Az algoritmusban meghívott függvények:}
\begin{itemize}
	\item \textit{recipe}: egy receptgráfot ad vissza
	\item \textit{select\_remove}: kiválaszt egy részproblémát valamilyen módszer alapján
	\item \textit{Feasible}: megmutatja, hogy a részprobléma a megadott időhorizonton belül megoldható-e
	\item \textit{IsLeaf}: megmutatja, hogy a részprobléma levélnek minősül-e ($J'==\emptyset$).
\end{itemize}

\begin{algorithm}[H]
\caption{Párhuzamos taszkvégrehajtást megvalósító algoritmus}
\label{parhuzamos}
\begin{algorithmic}[1]
\Procedure{Maxprofit}{TH,batch\_number}
	\State $profit^{cb}:= -\infty$
	\State $SOAA:= \emptyset$
	\State $S:= {(recipe(),I,J,\emptyset)}$
	\While{$S \neq \emptyset$}
		\State $PP:= select\_remove(S)$
		\If {$PP.Feasible(TH)$}
			\If{$PP.proift\_bound > profit^{cb}$}
				\If{$PP.IsLeaf()$}
					\State $profit^{cb}:=PP.profit\_bound$
				\Else
					\State $j:=select(J')$
					\ForAll	{$i \in I_j \setminus SOAA$}
						\State $SOAA_j: = SOAA_j\cup i$
						\State $G^i(N,A_1^i,A_2^i,w^i):= G(N,A_1,A_2,w)$
						\ForAll	{$i' \in \bigcup_{(i',j) \in {\cal A}} I_{i^i}^+  \setminus \{i\} $}
							\State $A_2^i:= A_2^i \cup \{(i',i)\}$				
						\EndFor
						\ForAll {$ i' \in I_i^+$}
							\State $w_{i,i'}^i:= t_{i,j}^{pr}$
						\EndFor
						\State $S:= S \cup (G^i(N,A_1,A_2^i,w^i),I,J',{\cal A} \cup \{(i,j)\})$
					\EndFor
					\If {$I_j \setminus \bigcup_{j \in J} SOAA_j \subseteq \bigcup_{j != j', j'\in J'} I_{j'}$}
						\State $S:= S \cup (G(N,A_1,A_2),I,J'\setminus\{j\},{\cal A})$
					\EndIf					
				\EndIf
			\EndIf
		\EndIf
	\EndWhile
	\State \Return $profit^{cb}$
\EndProcedure
\end{algorithmic}
\end{algorithm}
\chapter{Implementálás}
\section{S-gráf solver}
Az S-gráf megoldó szoftver egy a Pannon Egyetem, Műszaki Informatikai Kar, Rendszer- és Számítástudományi Tanszék által fejlesztett szoftver. C++ nyelven íródott, amely képes nagy teljesítményre, ami fontos a tudományos területeken. A szoftverben a C++ nyelv sajátosságai fedezhetőek fel, többek között az objektum orientált paradigma, valamint különböző a nyelvbe nem beépített tárolók, algoritmusok.

A szoftver szakaszos üzemű termelőrendszerek rövidtávú ütemezésével foglalkozik. A megoldó jelenleg a tárolási stratégiákat tekintve a NIS, UIS, UW és LW feladatokat támogatja. Ezek mellett lehetőség van AWS feladatok megoldására is. A célfüggvények közül a makespan minimalizáció, a throughput maximalizáció és a ciklusidő minimalizálás támogatott.

A szoftver felépítésében nagy szerepe van az objektum orientáltságnak az osztályok használatán keresztül. Ezek segítségével a megoldóban el vannak különítve a beolvasás, a végeredmény kiírás, valamint a különböző megoldó algoritmusokat végző részek, modulok. A program képes különböző formátumú fájlok beolvasására (Pl: xml, ods, csv), amelyek tartalmazzák a probléma megoldásához szükséges információkat. A bemeneti fájl alapján felépül a receptgráf, és ebből legenerálja a részproblémákat. A szoftver az eredmény képernyőn való megjelenítésén kívül fájlban is eltárolja azt, továbbá a Gantt diagram adatait karakteres formában is elmenti az említett fájlban, illetve lehetőség van arra is, hogy ezt a diagramot képfájlban mentse el. A megoldó használata környezeti változók segítségével történik. Ezek segítségével adható meg a bementi fájl, az eredményeket tároló fájl, a kívánt megoldó módszer meghatározása, időhorizont megadása, továbbá számos különböző beállítási lehetőség. Erre egy példa: $$-i\:input.ods -o\: output.txt -t\:2 -m\:eqbased$$ Az "-i" paranccsal adható meg a bemeneti, input fájl, az "-o" paranccsal pedig a kimeneti, output fájlt lehet megadni. A "-t" utasítás a maximálisan használható szálak megadására szolgál. A példában az utolsó paranccsal, az "-m" kapcsolóval, pedig a megoldó módszert tudjuk kiválasztani.

A megoldóban egyik legfontosabb szerepet tölti be a Branch and Bound, azaz a Korlátozás és Szétválasztás algoritmus. A döntések az ütemezés során azt reprezentálják, hogy melyik berendezés, melyik részfeladatot, taszkot végzi el, illetve ezek sorrendjét is megmutatják. A szétválasztás több módszerrel is végrehajtható, ezért a szoftver kialakítása révén létrehozhatóak, hozzáadhatóak új algoritmusok a megoldóhoz.
\section{Adatok beolvasása}
\subsection{Bemeneti fájl}
A program működéséhez szükséges adatokat ods kiterjesztésű fájlban lehet megadni. A solver mappa input almappájában megtalálható az \textbf{extended\textunderscore precedential.ods} fájl, amely segítségül szolgál ahhoz, ha új fájlt kívánunk létrehozni a saját feladatunkhoz tartozó adatokkal. Ez a fájl egy, a szoftver működtetéséhez alkalmas, korábban létrehozott fájl kibővített változata. Az abban megtalálható táblázatokhoz további oszlopok kerültek hozzáadásra, amelyben az új módszerhez szükséges adatok szerepelnek. Az új fájl a következő táblákat tartalmazza, jelezve az újonnan hozzáadott oszlopokat:
\begin{itemize}
  \item Product tábla tartalmazza a termékekkel kapcsolatos adatokat. Itt a változtatás revenue oszlop hozzáadása, amely az adott termék elkészítésével szerzett jövedelem.
  \item Equipment táblában találhatóak a berendezésekhez kapcsolódó információk. Újonnan került hozzáadásra az úgynevezett b\textunderscore capacity oszlop, amely az adott berendezés kapacitását mutatja meg.
  \item Precendence tábla, amely a gráfban szereplő éleket az él kezdő csomópontjának és végpontjának feltüntetésével szemlélteti. Két új, hasonló oszlop lett a táblázathoz illesztve, ezek a következőek:
  	\begin{itemize}
  		\item Az s\textunderscore percent oszlop, amely a táblázat task1 oszlopában szereplő részfeladathoz tartozó százalékot mutatja.
  		\item A d\textunderscore percent oszlop, amely a táblázat task2 oszlopában szereplő részfeladathoz tartozó százalékot mutatja.
  	\end{itemize}
  	\item A részfeladatok adatait tartalmazzó task tábla változatlanul került felhasználásra.
  	\item A Proctime táblában találhatóak meg azok az adatok, hogy melyik taszkot melyik berendezés tudja elvégezni, valamint, hogy mennyi idő szükséges ehhez. Ez szintén módosítás nélkül lett átemelve.
\end{itemize}
\subsection{Beolvasó függvény}
Az új módszer megoldásához szükséges adatok beolvasása, hasonlóan a fájl létrejöttéhez, már egy meglévő függvény kibővítésével valósul meg. Az értékek beolvasásáért a \textbf{RelationalProblemReader} osztály a felelős. Ennek feladata, hogy felépítse a receptgráfot, illetve ezt eljuttassa a \textbf{MainSolver} osztályhoz. A beolvasást végző függvények forráskódjai az \textbf{src\textbackslash lib} mappában megtalálható \textbf{relationalproblemreader.cpp} és \textbf{relationalproblemreader.h} fájlokban szerepelnek. Az ebben megtalálható függvények közül az adatok programba történő átemeléséhez a \textbf{ReadPrecedential()} függvényre van szükség, amely ki lett bővítve azzal, hogy az új adatokat is képes legyen feldolgozni. 

Az említett függvényben meghívásra kerül a \textbf{ParseEquipments(SGraph* graph)} eljárás, ami a fejlesztés során ki lett egészítve azzal, hogy vizsgálja meg, hogy a fájlban lévő equipment tábla rendelkezik-e b\textunderscore capacity nevű oszloppal.
Ha igen, akkor ellenőrzi, hogy a beolvasott érték negatív vagy sem. Abban az esetben, ha negatív, akkor a szoftver dob egy kivételt és a működés leáll, mivel csak nem negatív értékekkel oldható meg a probléma. Ellenkező esetben pedig az \textbf{SGraph} objektum \textbf{Equipment} objektumában, amely ki lett bővítve egy double típusú változóval az új adatok megőrzésének érdekében, eltárolásra kerül a beolvasott adat.

Következő változtatás az, hogy a fájlban megtalálható precedence tábla két új oszlopában (s\textunderscore percent, d\textunderscore percent) található értékeket el tudja a szoftver tárolni. Ezek az értékek az \textbf{SGraph} objektum \textbf{Recipe} objektumában tárolódnak. A tárolást úgy lehet elképzelni, mint egy mátrixot, ami azt mutatja meg, hogy az adott taszkból, melyik taszkba mutat él. A mátrixok mérete $N\times N$-es, ahol az N a taszkok számát jelenti. A \textbf{sourcePercents} azt testesíti meg, hogy annak a taszknak, amelyből az él indul, mekkora kapacitása megy az adott élen. A \textbf{demandPercents} pedig azt reprezentálja, hogy mekkora százalékot tud felvenni az a taszk, amelybe az él mutat.
\section{FlexBatchSchProblem osztály}
Ezen osztály feladata a branching, vagyis a szétválasztás elvégzése az ütemezés során. Megállapítja, hogy melyik taszkokhoz melyik berendezés vagy berendezések hozzárendelése szükséges. Másképpen fogalmazva azt kell meghatároznia, hogy melyik berendezésnek, melyik taszkokat kell elvégezni a lehető legnagyobb profit elérésének érdekében. Az osztály forráskódja megtalálható az \textbf{src\textbackslash solver} mappában a \textbf{flexbatchschproblem.cpp} és a \textbf{flexbatchschproblem.cpp} fájlokban. Ez az osztály egy származtatott osztály. A szülőosztálya az \textbf{EqBasedSchProblem} osztály, aminek szintén van egy ősosztálya az \textbf{SchProblem} osztály. Az új módszer lényege abban áll, hogy egy adott taszkhoz több berendezést is hozzá lehet rendelni, ezért az EqBasedSchProblem osztályban megtalálható Branching függvény az ott szereplő formában ehhez a megoldó módszerhez nem megfelelő. Az eddigi adattagok mellett, az átdolgozott kiválasztás módszer miatt, szükséges új adattagok bevezetése. Az első új adattag egy vectoron belüli vector segítéségével megvalósított mátrix, amely azt reprezentálja, hogy melyik berendezéshez melyik taszk lett már hozzárendelve. A másik új adattag pedig egy \textbf{IndexSet} típusú változó, amelyben azok a berendezések szerepelnek, amelyek még nincsenek ütemezve, azaz még képes elvégezni taszkokat.
\begin{lstlisting}[caption={FlexBatchSchProblem osztály adattagjai}]
class FlexBatchSchProblem: public EqBasedSchProblem{
protected:
	vector<vector<bool>> eqAssignedToTask;
    IndexSet sounEqs;
}
\end{lstlisting}
\subsection{MakeDecisions függvény}
Ennek a függvénynek a feladata az, hogy találjon egy berendezést, amelyhez még a probléma során még lehet legalább egy taszkot rendelni. Ha már nincs olyan berendezés amely még nem ütemezett a részproblémában akkor a függvény futása máris véget ér. Ha ez nem történik meg, akkor következik a berendezés keresése. Itt meg kell vizsgálni, hogy az éppen soron lévő berendezés szerepel-e azon berendezések halmazában, amelyeket még a részproblémák megoldásához igénybe lehet venni. A berendezések közti keresés addig tart, amíg nem talál egy olyat, amit legalább egy taszkhoz hozzá lehet rendelni. Ezt követően a probléma \textbf{Decision} típusú adattagjában ez a berendezés, illetve azok a taszkok amelyeket el tud végezni, kerülnek eltárolásra. Továbbá a függvényben kerül sor arra, hogy az említett adattagba beállítódjanak azok a taszkok, amelyeket csak az éppen kiválasztott berendezés képes elvégezni, valamint azok, amelyeket más berendezéshez vagy berendezésekhez is hozzá lehet rendelni. A döntést tartalmazó adattagban tárolásra kerül ezeken felül még az is, hogy az adott részproblémának mennyi gyereke lehet. Abban az esetben, ha olyan berendezés kerül kiválasztásra, amelyet csak olyan taszkokhoz lehet rendelni, amelyeket más berendezés is képes végrehajtani, akkor meg kell növelni a gyerekek számát, mert lehetséges olyan döntést hozni, hogy az adott berendezést semelyik lehetséges taszkhoz sem rendeljük hozzá.   
\subsection{Branching függvény}
Ez a függvény valósítja meg a szétválasztást a probléma megoldása során azaz, minden éppen aktuális részproblémára meghívja a Branch and Bound módszert megvalósító függvényt. Amennyiben az előző pontban már említett, \textbf{Decision} típusú adattagja nem üres, akkor lehetséges további döntéseket, hozzárendeléseket végezni. Az említett adattagban szerepel, hogy jelenleg melyik berendezésről kell dönteni, illetve szerepelnek azok a taszkok, amelyeket el tud végezni. Ezek közül a sorban az elsőt kiválasztja és megpróbálja az ütemezést végre hajtani az ősosztályban szereplő \textbf{Schedule} függvény meghívásával. Ha ez nem lehetséges, akkor nem felelt meg a feasible, megvalósíthatósági tesztnek. Ellenkező esetben az említett függvény hozzáadja a gráfhoz az ütemezési éleket, beállítja a hozzárendeléseket (az elvégzéshez szükséges időt), illetve újraszámolja a frissített ütemezési gráfhoz tartozó ProfitBoundot, a profit korlátot. Ezek után az osztály \textbf{eqAssignedToTask} adattagjában beállítja az imént a gráfban is beállított berendezés-taszk párost, hogy ezt később már ne lehessen újra egymáshoz rendelni. Ha mindezt követően a kiválasztott berendezést már csak egy taszkhoz lehet hozzárendelni, akkor a berendezést kivesszük a nem ütemezett berendezések halmazából. Abban az esetben, ha a kiválasztott berendezést már nem kívánjuk hozzárendelni taszkhoz, de van olyan taszk amit még el tudna végezni és ezt a taszkot más berendezés is el tudná végezni, akkor a berendezést kivesszük az ezt követő részproblémákból. Mindezen lépések után a \textbf{MakeDecisions} függvény segítségével ennek a részproblémának a gyerek problémájához hozunk döntést, valamint a korlátja is beállításra kerül.
\subsection{További metódusok}
Az \textbf{IsFeasible} függvénynek az a feladata, hogy elvégezze annak ellenőrzését, hogy az adott probléma megvalósítható vagy sem. Ehhez igénybe veszi az ősosztályban megtalálható ugyanezzel a névvel rendelkező függvényt. Ebben  megvizsgálásra kerül, hogy a gráfban található-e kör. A kör olyan egymáshoz csatlakozó élek sorozata, amelyben az élek és pontok egynél többször nem szerepelhetnek, és a kiindulási pont megegyezik a végponttal. Az új osztályban szereplő függvény ezt kibővíti azzal, hogy megvizsgálja, hogy a megadott időhorizonton belül megoldható-e a feladat. Ha e kettő feltétel közül valamelyiknek nem felel az adott feladat, akkor az éppen vizsgált részprobléma nem lesz megvalósítható a megadott feltételek mellett.

A \textbf{Bound} eljárás a korlátot állítja be. Mivel az elkészített módszer maximalizációra lett tervezve, a Solver keretrendszerben megtalálható további megoldó módszerekkel szemben, amelyek pedig minimalizálnak, ezért szükséges a negatív szorzó, hogy a korábban elkészített függvényekben megfelelő eredményeket lehessen elérni.

Megtalálható még egy egyszerű \textbf{IsComplete} névvel fellelhető függvény, amelynek csupán annyi a szerepe, hogy egy bool értéket ad vissza, ami azt mutatja meg, hogy a berendezések halmaza üres vagy nem. Az üres állapot azt jelenti, hogy az összes berendezés már ütemezett, vagyis nem szándékozunk, vagy nem lehet hozzá taszkokat rendelni. Ellenkező esetben pedig, legalább egyhez még lehet taszkot hozzárendelni.

Az osztályhoz tartozik három másoló függvény is a \textbf{FastClone}, a \textbf{Clone} és a \textbf{MakeCopy}. Az elsőnek említett függvény egyszerűen létrehoz egy új objektumot, aminek paraméterlistájában átadjuk a másolni szándékozott problémát. A Clone függvény az előbb említetthez képest abban tér el, hogy paraméterként true-t ad meg, ami azt mutatja meg, hogy a receptet is másolja vagy sem. A harmadik viszont meghív egy \textbf{CopyInto} névvel ellátott függvényt, ami adattagonként végzi a másolást. Ennél meg lehet adni, hogy teljes másolást végezzen, illetve megtartsa az eredeti problémában lévő döntéseket. További eltérés a három függvény között, hogy a \textbf{FastClone} elérhetősége proceted, ami azt jelenti, hogy csak a származtatott osztályai érik él, nem pedig bármely függvény. A másik két függvény viszont public, így azokat máshonnan, osztályon, származtatott osztályokon kívülről is el lehet érni. 

\section{SGraph osztály}
Az \textbf{SGraph} osztály egy olyan osztály, amely támogatja különböző műveletek elvégzését az S-gráfon. Többek között az ilyen feladatok közé tartozik az ütemezési élek hozzáadása, korlátok lekérdezése, leghosszabb út lekérdezése, valamint taszkok és berendezések közötti hozzárendelések megszüntetése. Az osztályban metódusok módosítására, valamint új függvények hozzáadására van szükség ahhoz, hogy képes legyen párhuzamos hozzárendelést megengedő feladatok elvégzésére. Két teljesen új függvény került hozzáadásra az \textbf{UpdateProfitBound} és az \textbf{UpdateProfitBoundFromTask}. Ezenkívül egy függvény nagyobb megváltoztatására is sor került, ez a függvény pedig a \textbf{MakeMultipleBatches}. Az előbb említett 3 metódus mindegyike az \textbf{sgraph.cpp} és \textbf{sgraph.h} fájlokban található meg. Ezek a fájlok az \textbf{src\textbackslash solver} mappában fellelhetőek. Az osztályt egy adattaggal kellett kibővíteni, amiben a taszkokhoz tartozó kapacitást lehet eltárolni, vagyis mekkora mennyiséget tud az adott taszk előállítani. Az adattag vector segítségével valósítja meg a tárolást, amelyben double típusú adatokat lehet elmenteni.
\subsection{MakeMultipleBatches függvény}
Ez a függvény abban az esetben játszik fontos szerepet, amikor legalább egy termékből egynél több darabot szeretnénk gyártani, vagyis a batch szám nagyobb lesz mint egy. Ilyenkor az adott termékhez tartozó minden taszk számát a program futása során annyira kell módosítani, amennyi terméket kell legyártani. Tekinthetünk úgy rá, hogy minden darab termékhez saját recept készül. A beolvasás eredetileg egy receptet készít el, vagyis minden taszkból csak egy szerepel. Ezen taszkok új száma alapján módosul már az 5.2.2. pontban említett \textbf{Recipe} osztályban lévő $N\times N$-es mátrixok mérete. A függvény a korábban meglévő módszerekhez szükséges adatok átdolgozásáról gondoskodik, csak az új adattagok módosításával kell foglalkozni, így a taszkok száma már megfelelő lesz, mikor a százalékokat tartalmazó mátrixot módosítani kell. Ezek a százalékok azt jelentik, hogy az él mekkora mennyiségekkel foglalkozik. A bemeneti fájlban megtalálható \textbf{s\textunderscore percent} oszlopban lévő adat azt mutatja meg, hogy az él, abból a taszkból, amelyből indul, onnan az ott gyártott mennyiség mekkora részét képes átvenni. A \textbf{d\textunderscore percent} oszlopban lévő adatok pedig, hogy az a taszk, amelybe az él befut, mekkora százalékát képes felvenni a mennyiségnek. 

Az eredetileg szereplő taszkok azonosítója megváltozik, mivel az új taszkokat nem csak hozzá adjuk azokhoz a következő azonosítóval. Az ugyanolyan taszkok egymást követő azonosítót kapnak, így a százalékokat tároló mátrixot is módosítani kell. Fontos dolog az, hogy csak az adott recepthez tartozó taszkok között lehet élt behúzni, nem lehet másik receptben szereplő taszkhoz hozzárendelni. Ezeket különböző feltételek bevonásával lehet megvalósítani. A megvalósítás úgy jött létre, hogy a mátrix  első sorának ellenőrzése kis mértékben eltér a további sorok átvizsgálásától. Ez az eltérése az over változóban mutatkozik meg, mégpedig úgy, hogy ez a változó ugyanazokat a taszkokat reprezentáló, különböző azonosítókat vizsgálja a feljebb lévő sorokban. A feltételek forráskódjai a következő részben tekinthetőek meg.
%A feltételek az ~\ref{MakeBatchFeltetelek} ábrán láthatóak.
%\begin{figure}[H]
%\begin{center}
%\includegraphics[scale=0.75]{MakeBatchFeltetelek}
%\caption{Éleken lévő százalékok beállításának feltételei}
%\label{MakeBatchFeltetelek}
%\end{center}
%\end{figure}

\begin{lstlisting}[language=C++, caption={FlexBatchSchProblem osztály adattagjai}, frame={single}]
if(return_graph->IsProfitMaximization()){
   vector<vector<double>> newSourcePercents(return_graph->GetRecipe()->GetTaskCount(), vector<double>(return_graph->GetRecipe()->GetTaskCount(),0));
   vector<vector<double>> newDemandPercents(return_graph->GetRecipe()->GetTaskCount(), vector<double>(return_graph->GetRecipe()->GetTaskCount(),0));
   bool wasAlreadyP = false;
   bool over = false;
   for(uint i = 0; i<return_graph->GetRecipe()->GetTaskCount(); i++){
      for(uint j = 0; j<return_graph->GetRecipe()->GetTaskCount(); j++){
         if(i==0){
            if(j>0 && tasksOldIds[j]!=tasksOldIds[j-1]){
               wasAlreadyP = false;
            }
            if(j>0 && tasksOldIds[j]==tasksOldIds[j-1] && GetRecipe()->getSourcePercent(tasksOldIds[i],tasksOldIds[j])!=-1 && !wasAlreadyP){
               wasAlreadyP = true;
            }
            if(!wasAlreadyP){
               newSourcePercents[i][j] = GetRecipe()->getSourcePercent(tasksOldIds[i],tasksOldIds[j]);
               newDemandPercents[j][i] = GetRecipe()->getDemandPercent(tasksOldIds[j],tasksOldIds[i]);
            }
            else{
               newSourcePercents[i][j] = -1;
               newDemandPercents[j][i] = -1;
            }
         }
         else{
            over = false;
            if(j>0 && tasksOldIds[j]!=tasksOldIds[j-1]){
               wasAlreadyP = false;
            }
            if(j>0 && tasksOldIds[j]==tasksOldIds[j-1] && GetRecipe()->getSourcePercent(tasksOldIds[i],tasksOldIds[j])!=-1 && !wasAlreadyP &&newSourcePercents[i][j-1]!=-1){
               wasAlreadyP = true;
            }
            for(uint k = 1; k<=i;k++){
               if(tasksOldIds[i]==tasksOldIds[i-k] && newSourcePercents[i-k][j]!=-1 ){
                  over =true;
               }
            }
            if(!wasAlreadyP && !over){
               newSourcePercents[i][j] = GetRecipe()->getSourcePercent(tasksOldIds[i],tasksOldIds[j]);
               newDemandPercents[j][i] = GetRecipe()->getDemandPercent(tasksOldIds[j],tasksOldIds[i]);
            }
            else {
               newSourcePercents[i][j] = -1;
               newDemandPercents[j][i] = -1;
            }
         }
      }
   }
   return_graph->GetRecipe()->SetSourcePercents(newSourcePercents);
   return_graph->GetRecipe()->SetDemandPercents(newDemandPercents);
}
\end{lstlisting}

A könnyebb átláthatóság érdekében az ~\ref{MakeBatchPelda} ábrán látható egy példa. Két darab A terméket akarunk legyártani. Az adatok fájlból való beolvasása során az A1-es taszk megkapja a 0. azonosítót, az A2 pedig az 1. sorszámot. Mivel kettőt gyártunk le, ezért meghívódik a \textbf{MakeMultipleBatches} függvény, és újra kiosztja az azonosítókat, miközben létrehozta kellő számban az eredetiről lemásolt taszkokat. 
\begin{figure}[H]
\begin{center}
\includegraphics[scale=0.7]{MakeBatchPelda}
\caption{Példafeladat}
\label{MakeBatchPelda}
\end{center}
\end{figure}
Az új sorrend a \ref{tab:table1} táblázatban látható. Zárójelben pedig látható, hogy az ~\ref{MakeBatchPelda} ábrán melyik sorban lévő recepthez tartozik.
\begin{table}[H]
  \begin{center}
  	\caption{A taszkokhoz tartozó azonosítók}
  	\captionsetup[table]{skip=10pt}
    \label{tab:table1}
    \begin{tabular}{|c|c|}
      \textbf{Taszk} & \textbf{Azonosító} \\     
      \hline
      A1 (első) & 0\\
      A1 (második) & 1\\
      A2 (első) & 2\\
      A2 (második) & 3\\
    \end{tabular}
  \end{center}
\end{table}
Azt nem lehet megengedni, hogy a 0. azonosítóval rendelkező taszk, a 3. azonosítóval rendelkező taszk között él keletkezzen, mert nem ez a kettő taszk tartozik egymáshoz. A példafeladatban látható, hogy egy él kezdő taszkjának, és annak a taszknak, amelybe érkezik, az azonosítójuk különbsége éppen annyi, amennyi terméket gyártani szeretnék. Jelen esetben kettő. 

\subsection{UpdateProfitBound függvény}
A függvény feladata, hogy kiszámolja az első S-gráfhoz tartozó korlátot, valamint minden egyes taszkhoz tartozó kapacitást is meghatározza. Legelső lépésben beállítja a taszkoknak az úgynevezett alap kapacitását. Ezt az alapján lehet meghatározni, hogy egyes berendezések, melyek a részfeladatot képesek elvégezni rendelkeznek kapacitással. Ezeket a bemeneti fájlból olvassa be a szoftver. Egy taszk kapacitását az összes, őt elvégezni képes taszk kapacitásának összege adja meg. Miután ez megtörtént a következő lépés a kezdő csomópontok, taszkok megkeresése. Ez 2 \textit{for} ciklus segítségével történik, amelyekben megvizsgáljuk, hogy az adott csomópont rendelkezik-e abba tartó, bementi éllel. Ha ilyen nincs akkor biztosak lehetünk benne, hogy az adott csomópont kezdő csomópont. Miután ezzel megvagyunk akkor megkeressük az előbb megtalált csomópontok szomszédjait. Ehhez egy \textbf{\textit{deque}} (double-ended queue), azaz kétvégű sort veszünk igénybe. Ennek előnye abban rejlik, hogy mind az elejéhez, mind a végéhez lehetséges elemet fűzni, illetve onnan eltávolítani. Ebbe a változóba tároljuk a kezdő csomópontok szomszédjait. Ismét két \textit{for} ciklus segítségével bejárjuk a taszkokat, amennyiben van köztük él, és még nem szerepel az adott taszk a \textbf{\textit{deque-ban}}, akkor beletesszük.

Mindezeket követően elérkezik az a rész, ahol a kapacitások felülvizsgálata következik. Egy \textit{while} ciklus segítségével minden \textbf{\textit{deque-ban}} szereplő elemet vizsgálunk addig, amíg az teljesen üressé nem válik. Először a \textbf{\textit{deque}} első elemét kivesszük belőle, majd egy \textit{for} ciklus segítségével ismét végighaladunk a taszkokon. Ha az éppen ciklusban lévő taszkból mutat él a \textbf{deque-ból} kivett taszkba, továbbá a taszknak, amiből az él indult, már korábban, a mostani függvény futása során, felül lett vizsgálva a kapacitása akkor lehet ellenőrizni a \textbf{\textit{deque-ból}} kivett taszk kapacitását. Ha az aktuálisan kiszámolt kapacitás nagyobb mint az eddigi, akkor a korábbi helyett az újat jelöljük ki a taszk kapacitásának. Ennek kiszámításhoz a következő képletet kell használni.
\begin{gather}
\scalebox{0.75}{$
\begin{align*}
\text{kapacitás}&= \text{előző taszk kapacitása}&*\frac{\text{előző taszkból felvett kapacitás mennyisége}}{\text{a mennyiég, amit az éppen vizsgált taszk feltud venni}}
\end{align*}$}	
\end{gather}
Abban az esetben, ha a kezdeti taszk még nem lett ellenőrizve, akkor nem lehet megvizsgálni az éppen kiválasztott taszkot, ezért visszatesszük a \textbf{\textit{deque}} végére. Ha viszont lehetséges volt és végbe is ment az adott taszk felülvizsgálata, akkor megkeressük ennek a csomópontnak a szomszédjait és hozzáfűzzük a \textbf{\textit{deque}} végéhez.

A függvény utolsó szakaszában történik meg a profit korlát meghatározása, kiszámolása. A korlátot azoknak a taszkoknak a kapacitása adja meg, amelyek a termékek előtti utolsó részfeladatok. Ezek megtalálása úgy történik, hogy \textit{for} ciklus segítségével bejárjuk a termékeket, valamit a taszkokat is. Ha valamelyik taszkból indul él egy termékbe, akkor a taszk kapacitását megszorozzuk a termékből származó jövedelemmel.

\subsection{UpdateProfitBoundFromTask függvény}
Ez a függvény feladatában hasonlít az előző pontban bemutatott \textbf{UpdateProfitBound} metódushoz. A taszkokhoz tartozó kapacitást, valamint a korlátot kell meghatároznia. Különbséget abban lehet felfedezni, hogy ennek a függvénynek nem kell a teljes S-gráfot bejárnia, az összes kapacitást nem szükséges újraszámolnia, hanem csak a paraméterben megadott taszkokhoz, és az ezt követő taszkokhoz tartozó kapacitásokat kell újraszámolnia. Az ezt követő taszkokat úgy kell értelmezni, hogy a megadott taszk szomszédjait, valamint azoknak a szomszédjait (így tovább egészen addig, amíg létezik egy szomszéd taszk) kell átvizsgálni és szükség esetén megváltoztatni, módosítani a kapacitásukat.

Az előző pontban bemutatott függvénytől eltérően nem az S-gráf bemeneti node-jait, csomópontjait reprezentáló taszkokat kell először megkeresni, hanem a paraméterlistában átadottat, valamint annak közvetlen szomszédjait. Ehhez is a \textbf{\textit{deque-t}} veszünk igénybe. Legelsőnek az átadott taszk kerül bele, majd \textit{for} ciklus segítségével megkeresi annak szomszédjait, és ezeket is a \textbf{\textit{deque}} végéhez hozzáfűzi. Ezek után meg kell keresni a többi olyan taszkot is, amelyeknek a kapacitását újra át kell vizsgálni, és ha szükséges módosítást végezni. Azt követően, hogy a két végű sor tartalmazza az összes átvizsgálandó taszkot megtörténik a tényleges kapacitásmódosítás. Itt \textit{while} ciklus felhasználásával addig történik az ellenőrzés, amíg teljesen üressé válik a \textbf{\textit{deque}}. Ebből kivételre kerül a legelső elem, és megvizsgáljuk, hogy van-e ebbe tartó él, vagyis az S-gráf kezdő csomópontja vagy sem. Későbbiekben lesz szerepe ennek. Az éppen vizsgált taszk kapacitását átállítjuk nullára, majd a még hozzárendelhető berendezések kapacitásának összegét megkapja, mint új értéket. Amennyiben a taszknak nincs bejövő éle, akkor az imént meghatározott kapacitása megmarad, nincs szükség további ellenőrzésekre. Ellenben, ha van bemenő él, akkor még további feltételekre meg kell vizsgálni. Ha az a taszk, amelyből az él érkezik még nem ellenőrzött, akkor nem lehetséges a mostani taszk kapacitásának pontos meghatározása sem, ezért visszakerül a \textbf{\textit{deque}} végére. Azonban ha ellenőrzött a vizsgált taszkot megelőző részfeladat, akkor az előző pontban feltüntetett képlet szerint ki kell számolni a kapacitást. Ha ez nagyobb, mint a beállított akkor ezt kapja meg a taszk új értékként. Ellenkező esetben pedig marad a már meglévő érték. Ezeket követően szükséges még egy \textit{for} ciklus segítségével végigmenni a taszkokon, annak érdekében, hogy ha létezik megtalálja az összes szomszédját az imént vizsgált taszknak. Ha talált ennek megfelelő taszkot akkor a \textbf{\textit{deque}} végéhez hozzáadjuk.

Utolsó lépés a függvényben a profit korlát kiszámítása. Ez teljes mértékben megegyezik az előző pontban szereplő függvény befejező lépésével. Az S-gráfon a termék előtt szereplő utolsó részfeladat kapacitása szükséges a korlát kiszámításához. Azért, hogy megtaláljuk ezt a taszkot szükség van arra, hogy két darab \textit{for} ciklus bejárja a termékeket és a taszkokat. Ha megtalálta akkor annak kapacitása és a terméken szerzett jövedelem szorzata megadja a korlátot.

\subsection{Egyéb új metódusok}
Az \textbf{IsProfitMaximization} függvény megadja, hogy a megoldó szoftver indításakor az új módszer került meghívásra parancssori paraméterek által. Olyan esetekben kerül meghívásra, amelyeket csak abban az esetben kell végrehajtani, ha az új $-$ taszkok párhuzamos végrehajtására alkalmas $-$ módszer lett meghívva. Például a futás végén a fájlba írásnál kapacitásokat csak ennél a módszernél akarunk kiírni.

A másik egyszerűbb függvény a \textbf{GetTaskCapacity}. Paraméterként egy részfeladat azonosítóját várja, és ez alapján visszaadja az adott taszkhoz tartozó kapacitást.

\section{Argumentum hozzáadás}
Új módszer elkészítése miatt szükség volt új argumentumok hozzáadására a meglévők mellé. Az argumentumokat az \textbf{arguments.cpp} fájlban lehet elérni. Ez a fájl az \textbf{src\textbackslash solver} mappában található. Két argumentum került hozzáadásra. Az első \textbf{\textit{flexbatch}}, amely a \textbf{method} kapcsolóhoz tartozik, ezzel a megoldó módszert lehet kiválasztani. A második új argumentum a \textbf{\textit{profit\textunderscore max}}, amely a \textbf{obj} kapcsolóhoz tartozik, ez pedig a célfüggvényt reprezentálja.

\section{Megoldás fájlba írása}
Az új módszerrel kapcsolatos kapacitások kezelése eddig nem volt része a solver megoldó szoftvernek. Ez alól a fájlba történő kiírásuk sem kivétel, emiatt szükség volt a már meglévő kiírást elvégző függvény módosítására. A szóban lévő függvény a \textbf{WriteText}, amely \textbf{SGraph} típusú változót vár paraméterként. A függvény \textbf{solutionwriter.cpp} fájlban található míg a deklarációja a \textbf{solutionwriter.h} fájlban helyezkedik el. Ezeket a fájlokat a \textbf{src\textbackslash lib} mappában találhatjuk meg. 

Az eredetileg meglévő függvény a fájlba két elkülöníthető rész kiírását végezte. Az első az volt, amely megmutatta az éleket. Ebbe beletartozik mind a recept, mind az ütemezési élek csoportja. Ezenfelül még megjelenik itt egy időérték, amely azt mutatja, hogy az adott taszkot, amelyből az él kiindul mennyi idő alatt lehet befejezni, elvégezni. Továbbá a boundot, korlátot is itt írja ki a fájlba a függvény. A második fele pedig az elvégzett feladathoz tartozó Gantt diagramot jeleníti meg karakteres formában. Ez a rész megadja, hogy melyik taszkot, melyik berendezés végezi el és, hogy ez mikor történik.

Az említett két rész közé került az általam elkészített kapacitások kiírására szolgáló rész. Pontosabban, mivel az új módszer kapacitás és jövedelem szorzataként adja meg a korlátot ezért, a rész a kapacitások után jelenik ezentúl meg. Először a taszkok kapacitása kerül kiírásra, ezeket követik a termékek, amelyekhez az utolsó taszk kapacitása és a termékből származó jövedelem adja meg az értéket. A korlát kiszámítása a termékekhez tartozó értékek összeadásával történik.
\chapter{Tesztelés}
Ebben a fejezetben az elkészített megoldó algoritmus tesztelését mutatom be, hogy az megfelelően, hiba nélkül képes a feladatokat megoldani.
A tesztelés a következő konfigurációval rendelkező számítógéppel történt:
\begin{itemize}
	\item Processzor: Intel i5-7200, 2,50 Ghz
	\item 8 GB RAM
	\item Operációs rendszer: Windows 10
	\item Fejlesztésnél használt szoftverek:
		\begin{itemize}
			\item Qt Creator 4.7.1
			\item Qt 5.11.2
			\item Boost Libraries 1.68.0
			\item Microsoft Visual C++ Compiler 15.0 
		\end{itemize}		 
\end{itemize}

Az S-gráf megoldó szoftvert parancssori paraméterek segítségével lehet működtetni. A különböző megoldó módszereket, amelyek megtalálhatóak a szoftverbe implementálva, más és más kapcsolók segítéségével lehet elérni, meghívni. Ezeknek listája megtalálható a \textbf{solver} mappában lévő \textbf{README.md} fájlban. Az általam megvalósított módszerhez a következő kapcsolókat mindenképpen használni kell, hogy a program hiba nélkül fusson és elvégezze az ütemezést:
\begin{itemize}
	\item \textbf{-i extended\textunderscore precedential.ods:} A bemeneti fájl elérési útvonalát kell megadni ezzel.
	\item \textbf{-o output.txt:} A kimeneti fájl elérési útvonala. Két fajta kiterjesztésű fájlt lehet megadni: \textbf{TXT} és \textbf{PNG}. Előbbi esetében a fájlba kerülnek az élek, mind a receptek, mind az ütemezési élek, valamint, hogy mennyi ideig tart a részfeladat befejezése, amelyből ezek kiindulnak. Ezenfelül a Gantt diagram karakteres formában is megjelenik. PNG kiterjesztésű fájl megadása esetén pedig kirajzolásra kerül egy Gantt diagram. Azonban, ha mindkét kiterjesztésű fájlra szükség van, akkor erre a \textbf{-g} kapcsoló segítségével van lehetőség. Ehhez a kapcsolóhoz kell a PNG kiterjesztésű fájl nevét megadni.
	\item \textbf{-m flexbatch:} Ezzel a kapcsolóval a megoldó módszert lehet kiválasztani. A \textit{felxbatch} határozza meg, hogy az általam megvalósított algoritmus kerüljön meghívásra.
	\item \textbf{--timehor:} Időhorizont megadása történik ezzel a kapcsolóval.
	\item \textbf{--obj profit\textunderscore max:} A célfüggvény kiválasztása, ebben az esetben az újonnan a megoldó szoftverhez hozzáadott profit maximalizálás kerül kiválasztásra.
	\item \textbf{--precycle off:} A precycle az ütemezés gyorsításra szolgál, mégpedig úgy, hogy előre lefut, és kört keres a gráfban. Az új módszer esetén hibásan működik, nem összeegyeztethető azzal, ezért szükséges a kikapcsolása.
	\item \textbf{--nopresolvers:} Hasonlóan az előzőhöz a presolver is az ütemezést gyorsítja, de nem egyeztethető össze az új megoldó módszerrel, emiatt kell mindenképpen inaktívvá tenni.
\end{itemize}

A ~\ref{tesztFeladat} ábrán látható mintafeladat alapján kerül bemutatásra a megoldó módszer. Két termékhez tartozó receptet látunk, amely a taszkokat, az ezeket megvalósítani képes berendezéseket, és az ehhez szükséges időt tartalmazza. Az éleken megfigyelhető még, hogy a kapacitások hány százaléka kerül továbbadásra, illetve a következő taszk által felvételre.
\begin{figure}[H]
\begin{center}
\includegraphics[scale=0.8]{tesztFeladat}
\caption{Tesztfeladat}
\label{tesztFeladat}
\end{center}
\end{figure}

\section{A tesztfeladat megoldása}
A módszer sajátossága, hogy a receptélekkel az egyes taszkok kapacitásának meghatározott százaléka hasznosítható. Ezeket a százalékokat a bemeneti fájlban kell megadni. A példafeladat a~\ref{bemenet1} ábrán megtekinthető bemeneti adatokkal rendelkezik. Az időhorizont a bemutatott mintafeladat során 15. Látható, hogy az egyes részfeladat által lehetséges kapacitásoknak nem a teljes mennyisége kerül tovább a következő taszkhoz. Az, hogy kisebb mennyiség kerül felhasználásra nagy mértékben befolyásolja a korlátot. Emellett az ütemezés megoldását is befolyásolja, mert emiatt lehetséges, hogy bizonyos berendezések nem végezhetik el az adott taszkot.

Az egyszerűség kedvéért a példában csak 2 darab termék receptje szerepel. Az A termék legyártásából származó jövedelem egy egység, a B termék esetén pedig 2 egység. A \textit{precendce} táblában látható, hogy a taszkok által elérhető mennyiségek nem 100 százalékban kerülnek további felhasználásra. Például az A1 és A2 taszkok esetében, az A1-es taszk mennyiségének 80 százaléka kerül továbbításra, illetve ennek a már kisebb mennyiségnek a 95 százalékát veszi fel az A2-es részfeladat. A \textit{proctime} táblán megtalálhatjuk, hogy melyik taszkot, melyik berendezés tudja elvégezni, valamint ezt mennyi idő alatt teszi.
 
\begin{figure}[H]
\begin{center}
\includegraphics[scale=0.7]{bemenet1}
\caption{A tesztfeladat bemeneti adatai}
\label{bemenet1}
\end{center}
\end{figure}

A feladat megoldását tartalmazó TXT kiterjesztésű fájlt három darab elkülöníthető részre tudjuk felosztani. Az első része látható a~\ref{eredmeny1} ábrán. Az ábra elején az éleket találhatjuk meg, mégpedig olyan formában, hogy melyik taszkból melyik taszkba mutat. Továbbá láthatóak időértékek, amelyek megadják, hogy az élt megelőző részfeladatot mennyi idő alatt lehet elvégezni. Ez a rész tartalmazza mind a receptéleket, mind az ütemezési éleket. Úgy tudjuk ezeket elkülöníteni, hogy amelyikhez 0 időérték van rendelve, azok tartoznak az ütemezési élek csoportjához. Mivel az egyik termékből, nevezetesen az B-ből, több mint egy darabot gyártunk ezért megkülönböztetjük az egyes receptekhez tartozó feladatokat. Láthatunk B1-et és B1\textunderscore 2-t. Ugyanolyan típusú feladatról beszélünk, de különböző recepthez tartoznak, ezért kell megkülönböztetni egymástól ezeket.
 
\begin{figure}[H]
\begin{center}
\includegraphics[scale=1]{eredmeny1}
\caption{A megoldást tartalmazó fájl első része}
\label{eredmeny1}
\end{center}
\end{figure}

A második részben a taszkok és a hozzájuk tartozó kapacitások, valamint a termékekhez tartozó kapacitások és a termékek előállításából származó jövedelem szorzata található. Ezeket követően szerepel a bound, a korlát, ami az adott feltételek és adatok mellett 482 lett. Ezt úgy kapjuk meg, hogy a termékekhez tartozó értékeket összeadjuk. Jelen példa esetében 3 érték kerül összeadásra, ezek a következők: A, B és B\textunderscore 2. Az A termékből származó jövedelem 50, a B és a B\textunderscore 2 termékek esetén 216. Ezeket összeadva kijön a 482. Ezek az értékek a~\ref{eredmeny2} ábrán megtekinthetőek.

\begin{figure}[H]
\begin{center}
\includegraphics[scale=1.1]{eredmeny2}
\caption{A megoldást tartalmazó fájl második része}
\label{eredmeny2}
\end{center}
\end{figure}

Az utolsó szakaszban a Gantt diagram karakteres formában található meg. Látható, hogy vannak olyan taszkok, amelyeket több berendezés is el tud végezni. A \ref{eredmeny3} ábrán látható ez.

\begin{figure}[H]
\begin{center}
\includegraphics[scale=1]{eredmeny3}
\caption{A megoldást tartalmazó fájl harmadik része}
\label{eredmeny3}
\end{center}
\end{figure}

A \ref{kimenet} ábrán pedig a PNG kiterjesztésű fájl látható. Azonos színnel jelölt taszkok tartoznak egy termékhez. Ha az egyik termékből több példány készül, akkor a termékhez tartozó taszkok végéhez illesztett szám mutatja meg, hogy melyik termékhez tartoznak ezek. A diagramon könnyen észrevehető az új módszerben lévő újítás. A zölddel jelzett B3\_2-es taszk, valamint a kék színű B3-as taszk két berendezéshez, nevezetesen az E2 és az E3 berendezéshez is hozzá lett rendelve. 
\begin{figure}[H]
\begin{center}
\includegraphics[scale=0.5]{kimenet}
\caption{A solver által elkészített Gantt diagram}
\label{kimenet}
\end{center}
\end{figure}

\section{A régi és az új megoldó összehasonlítása}
Ebben az alfejezetben az eddig meglévő throughput maximalizáló megoldót hasonlítom össze az általam létrehozottal. Ehhez \ref{tesztFeladat} ábrán látható feladatot veszem igénybe. Mivel ebben a feladatban a termékek változó batch mérettel rendelkeznek, ezért a régi megoldó módszerhez szükség van a 3. fejezetben bemutatott diszkretizálás végrehajtására. Ezt elvégezve eredményül azt kapjuk, hogy az A termék gyártása esetén mindenképpen 50 jövedelemre tudunk szert tenni. B termékhez viszont 4 különböző recept jött létre, amelyek mind eltérő jövedelmet biztosítanak. Ezek alapján felrajzolható a 4 gráf, amelyek a \ref{regi_grafok} ábrán láthatóak.
\begin{figure}[H]
\begin{center}
\includegraphics[scale=0.5]{regi_grafok}
\caption{Régi módszer esetén a B termékhez tartozó 4 gráf}
\label{regi_grafok}
\end{center}
\end{figure}
Az ábra bal oldalán szereplő gráfok megegyeznek a példafeladaton szereplő gráffal. Eltérés csak a taszkokat elvégezni tudó berendezésekben van. Azokat a  taszkokat, amelyeket 2 darab berendezés is el tud végezni, úgy kell figyelembe venni, hogy vagy az egyik vagy a másik berendezés végezi majd el az ütemezés után. A jobb oldalon látható, hogy a B3-as taszkból kettő darab lesz. Ezek az estek az jelentik, hogy mindkét berendezés elvégzi ezt a részfeladatot. A régi módszer esetén úgy tekintünk a termékekre, mintha 5 különböző termék lenne. Ezt az 1 darab A termék receptje és a 4 darab különböző B termék receptje adja meg. Ezek alapján elkészíthető a régi megoldóhoz szükséges bemeneti fájl, amely a következő ábrán látható. 

\begin{figure}[H]
\begin{center}
\includegraphics[scale=0.6]{regi_bemenet}
\caption{A régi megoldó bemeneti adatai}
\label{regi_bemenet}
\end{center}
\end{figure}

Látható, hogy ilyen típusú feladat esetében jóval több adat van a fájlban. Ennek az oka a több recept, ami több taszkot és élt jelent. Itt az 5 darab termék miatt egy 5 dimenziós teret lehet elképzelni. Az új megoldó módszernél ez csak 2 dimenziós tér lesz. Láthatjuk, hogy kevesebb dimenziót kell megvizsgálni, viszont egy dimenzión belül több számítást végez el az algoritmus a különböző hozzárendelési lehetőségek miatt. 

Az algoritmus lefutásának végeztével azt várjuk, hogy a régi és az új megoldó ugyanazt az eredményt szolgáltassa. Pontosítva a jövedelem mennyisége és a legyártott termékeknek meg kell egyezniük, a taszkok ütemezésétől azonban ezt nem várjuk el. A \ref{regi_kimenet} ábrán látható a régi megoldómódszer által elkészített Gantt diagram. Összehasonlítva a \ref{kimenet} ábrán látható diagrammal meg lehet állapítani azt, hogy mindkét esetben 2 B termék és 1 darab A terméket lehet legyártani a megadott időhorizonton belül. A másik fontos dolog, hogy a jövedelem is megegyezzen. Az új megoldónál ez 482 volt. A régi által nyújtott eredményben az A termék jövedelme 50, a két B termékből pedig olyanokat gyárt, amelyeknek 216 a jövedelme. Ezeket összeadva, 2 * 216 + 50, itt is kijön a 482. Az ütemezés során a taszkok sorrendje a két megoldó módszer esetén eltér, azonban ez nem hiba. Lehetnek olyan feladatok, ahol ez is teljes mértékben megegyezik.
\begin{figure}[H]
\begin{center}
\includegraphics[scale=0.6]{regi_kimenet}
\caption{A régi megoldó által nyújtott Gantt diagram}
\label{regi_kimenet}
\end{center}
\end{figure}

\section{A megoldó módszerek sebessége}
A következő táblázatban a megoldó módszerek sebességének összehasonlítása látható. A régi módszerbe már több, különböző gyorsítási megoldás be van építve. Ezek közé tartozik a korábban már említett precycle és presolver. Az új módszer esetén ezek a gyorsítások jelen állapotban nem működnek. Egy másik kiemelt fontosságú gyorsítás a régi megoldó módszer esetében a revenue line. Ez alsó korlátként szolgál az egyes konfigurációkból származó jövedelmek összehasonlításához. Ennek segítségével, feladattól függően, nagy számú konfiguráció eltávolítható a keresési térből, így jelentősen csökkenthető a számítások mennyisége az ütemezés során. Az új módszer használata során azonban ezt nem vehetjük igénybe, mert nem egy egyenesen szerepelnek azok a konfigurációk, amelyek azonos jövedelemmel rendelkeznek. 

\begin{table}[H]
	\begin{center}
	\caption{Különböző módszerekkel megoldott feladatok összehasonlítása}
	\captionsetup[table]{skip=10pt}
	\label{teszteredmenyek}
\begin{tabular}{|l|l|c|c|c|c|}
\hline
                                                 & \multicolumn{1}{c|}{Megoldó módszer} & Futás ideje          & Time Horizon & Legyártott termék & Profit \\ \hline
\multicolumn{1}{|c|}{\multirow{9}{*}{\rotatebox{90}{Feladat 1}}} & Új                                   & 1,057 mp             & 15           & 1A+2B             & 482    \\ \cline{2-6} 
\multicolumn{1}{|c|}{}                           & Régi                                 & 0,149 mp             & 15           & 1A+2B             & 482    \\ \cline{2-6} 
\multicolumn{1}{|c|}{}                           & Régi (gy. n.)                        & 0,185 mp             & 15           & 1A+2B             & 482    \\ \cline{2-6} 
\multicolumn{1}{|c|}{}                           & Új                                   & 94,964 mp            & 20           & 1A+3B             & 698    \\ \cline{2-6} 
\multicolumn{1}{|c|}{}                           & Régi                                 & 8,17 mp              & 20           & 1A+3B             & 698    \\ \cline{2-6} 
\multicolumn{1}{|c|}{}                           & Régi (gy. n.)                        & 13,553 mp            & 20           & 1A+3B             & 698    \\ \cline{2-6} 
\multicolumn{1}{|c|}{}                           & Új                                   & \textgreater 8000 mp & 25           & 4B                & 864    \\ \cline{2-6} 
\multicolumn{1}{|c|}{}                           & Régi                                 & 705,439 mp           & 25           & 4B                & 864    \\ \cline{2-6} 
\multicolumn{1}{|c|}{}                           & Régi (gy. n.)                        & 1175,11 mp           & 25           & 4B                & 864    \\ \hline
\multirow{9}{*}{\rotatebox{90}{Feladat 2}}                      & Új                                   & 0,371 mp             & 15           & 2C+1D             & 188    \\ \cline{2-6} 
                                                 & Régi                                 & 0,07 mp              & 15           & 2C+1D             & 188    \\ \cline{2-6} 
                                                 & Régi (gy. n.)                        & 0,113 mp             & 15           & 2C+1D             & 188    \\ \cline{2-6} 
                                                 & Új                                   & 15,486 mp            & 20           & 3C+1D             & 252    \\ \cline{2-6} 
                                                 & Régi                                 & 1,306 mp             & 20           & 3C+1D             & 252    \\ \cline{2-6} 
                                                 & Régi (gy. n.)                        & 3,201 mp             & 20           & 3C+1D             & 252    \\ \cline{2-6} 
                                                 & Új                                   & 741,29 mp            & 25           & 3C+3D             & 360    \\ \cline{2-6} 
                                                 & Régi                                 & 58,28 mp             & 25           & 3C+3D             & 360    \\ \cline{2-6} 
                                                 & Régi (gy. n.)                        & 222,679 mp           & 25           & 3C+3D             & 360    \\ \hline
\multirow{6}{*}{\rotatebox{90}{Feladat 3}}                       & Új                                   & 14,78 mp             & 10           & 2H+2I             & 86     \\ \cline{2-6} 
                                                 & Régi                                 & 0,374 mp             & 10           & 2H+2I             & 86     \\ \cline{2-6} 
                                                 & Régi (gy. n.)                        & 0,381 mp             & 10           & 2H+2I             & 86     \\ \cline{2-6} 
                                                 & Új                                   & 44,693 mp            & 15           & 4I                & 94     \\ \cline{2-6} 
                                                 & Régi                                 & 0,985 mp             & 15           & 4I                & 94     \\ \cline{2-6} 
                                                 & Régi (gy. n.)                        & 0,806 mp             & 15           & 4I                & 94     \\ \hline
\end{tabular}
\end{center}
\end{table}

Ugyanaz a feladat 3 módszerrel került megoldásra. A régivel, amelyben szerepelnek a gyorsítások, ismét a régi megoldóval (régi gy. n.), de itt a precycle és a presolver gyorsítások nélkül, illetve az elkészített új megoldó módszerrel. Látható, hogy a régi módszerek gyorsabban találjak meg a feladat legjobb megoldását, mint az új a jelenlegi állapotában. Jövőbeni tervek közé tartozik a meglévő gyorsítások átalakítása, hogy kompatibilisek legyenek az új módszerrel is. Ezek megvalósítása után nagy csökkenés várható az új megoldó módszer futási idejében. 
\chapter{Összefoglalás}
A dolgozatomban a throughput maximalizálás szakaszos üzemű rendszerekben témakörrel foglalkoztam.
Először az ütemezéssel kapcsolatos irodalmat tanulmányoztam, amelynek során megismerkedtem az ipari környezetben használt ütemezési megoldó módszerekkel.
Ezek közül az S-gráf keretrendszer lett az, amely a dolgozatom alapját képezi.
Ezt követően megismerkedtem a makespan minimalizálás és a throughput maximalizálás algoritmusával.
Ezek megismerése adta meg az alapot, hogy elkészülhessen a flexibilis batch mérettel rendelkező feladatok megoldására alkalmas módszer.
Ez egy már meglévő S-gráf keretrendszert megvalósító szoftverbe történő implementálással valósult meg.
Az implementáció megvalósítását követően a bővített rendszert tesztelésnek vetettem alá.
A tesztelés eredményei azt mutatták, hogy a vizsgált feladatokra helyesen működik a program, mivel a régi megoldó módszerrel megegyező eredményeket szolgáltatott.
Ennek az új módszernek a segítségével idő spórolható meg az előfeldolgozó lépések során, mert már nem szükséges a diszkretizációs folyamat elvégzése.
Azonban a futási sebessége nem hozott minden feladat esetén jobb eredményeket, mint a régi megoldó módszer.
A jövőbeni tervek közé tartozik ennek az időnek a csökkentése.

Már felmerült több különböző gyorsítási ötlet.
Jelenleg a megoldó több batch esetén megvizsgál minden részproblémát, azokat is,
amelyekben a taszkok sorrendje megegyezik, csak egy másik, ugyanolyan termékhez tartoznak.
Ha már az első ilyen megvizsgált részprobléma során kiderül, hogy nem megvalósítható, akkor nagy mennyiségű számítást lehetne megspórolni, ha a többi részproblémát már a megoldó nem vizsgálná meg.
Ezen kívül érdemes lehet különböző stratégiákat megvizsgálni a berendezések kiválasztására.
Elképzelhető, hogy a jelenlegi kiválasztási stratégiánál található egy jobb.
Jelenleg a profitmaximalizáló algoritmus első futásakor a profit értéke 0.
Amennyiben a keresés a tengelyek mentén megtalált addigi legjobb profitról indulna, akkor elképzelhető, hogy bizonyos számú konfiguráció ütemezése feleslegessé válna, mert ebben az esetben az adott konfiguráció legjobb megoldása is kisebb profitot szolgáltatna, mint a tengelyeken megtalált legjobb megoldás.

\bibliographystyle{ieeetr}
\bibliography{TDK}
\appendix
\chapter{Tesztesetek}

\begin{figure}[H]
\begin{center}
\includegraphics[scale=0.85]{teszteset8}
\caption{Nyolcadik teszteset bemeneti adatai}
\label{teszteset8}
\end{center}
\end{figure}

\begin{table}
	\begin{center}
	\caption{Második példafeladat teszteseteinek második fele}
  	\captionsetup[table]{skip=10pt}
  	\label{tab:table3}  	
  	\begin{sideways}  	  	
  		\begin{tabular}{|c|c|c|c|c|c|c|c|c|}
  		\hline
		Fájl név & Idő horizont & A termék & B termék & Kapacitások & Hozzárendelések & Bevétel & Megoldás ideje & Gantt \\
		\hline
		teszt06 & 25 & 2 db & 1 db & \makecell{A1: 100 \\ A1\_2: 100\\A2: 40\\A2\_2: 40\\A3: 40\\A3\_2:40\\B1: 100\\B2: 100} & \makecell{E1: A1, A1\_2,\\A2, A2\_2 \\B1, B2 \\E2: A1, A1\_2,\\ A3, A3\_2, \\ B1, B2} & 1300 & 0,162 sec & er06 \\
		\hline	
		teszt07 & 20 & 2 db & 1 db & \makecell{A1: 60\\A1\_2: 100\\A2: 40\\A2\_2: 40\\A3: 20\\A3\_2:20\\B1: 60\\B2: 60} & \makecell{E1: A1\_2, A2,\\A2\_2 \\ E2: A1, A1\_2,\\ A3, A3\_2, \\ B1, B2} & 700 & 0,844 sec & er07 \\
		\hline	
		teszt08 & 25 & 2 db & 1 db & \makecell{A1: 100 \\ A1\_2: 100\\A2: 40\\A2\_2: 40\\A3: 20\\A3\_2:20\\B1: 100\\B2: 100} & \makecell{E1: A1, A1\_2,\\A2, A2\_2 \\B1, B2 \\E2: A1, A1\_2,\\ A3, A3\_2, \\ B1, B2} & 900 & 0,220 sec & er08 \\
		\hline	
		\end{tabular}
	\end{sideways}
	\end{center}
\end{table}

\end{document}