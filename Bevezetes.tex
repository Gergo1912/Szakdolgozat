\chapter{Bevezetés}
Az ütemezés feladatával az élet számos területén találkozunk, kezdve az egyszerű, hétköznapi problémáktól, mint például egy napon elvégzendő feladataink sorrendjének beosztásával kezdve, professzionális sportcsapatok heti edzésprogramjának kialakításán át, egészen az ipari üzemek működéséig, ahol a rendelkezésre álló berendezésekhez kell rendelni az előállítani kívánt termékeket. Bár az életünk különböző területein fellépő problémák különböznek egymástól, bizonyos mértékben hasonlóság is fellelhető közöttük. Különbség lehet az ütemezési feladatok célfüggvénye, továbbá az adott ütemezési probléma lehet online, offline, illetve sztochasztikus vagy determinisztikus. Minden fellépő probléma esetében az a cél, hogy az elvégzendő feladatokat a rendelkezésre álló erőforrások között megosszuk oly módon, hogy adott intervallumon belül a lehető legjobb megoldást kapjuk, miközben a folyamat során megjelenő, fellépő korlátokat betartjuk, azokat nem sértjük meg. Az ipari ütemezés két leggyakoribb célja a makespan minimalizálás, és a throughput maximalizálás. A makespan kifejezést az idő minimalizálására, a throughput kifejezést pedig a profit maximalizálásra alkalmazzák az irodalomban.

Az ipari gyártási folyamatok ütemezésére különböző módszerek léteznek már. Ezek közé tartoznak a MILP megoldó módszerek, amelyek a lineáris programozáson alapulnak. Továbbá ide sorolható az időzített automaták, Petri hálók, valamint a S-gráf megoldómódszer, amely a munkám során a legfontosabb szerepet tölti be a felsorolt megoldó módszerek közül. Dolgozatom második fejezetében ezek kerülnek bemutatásra. A harmadik részben a probléma kerül definiálásra. A negyedikben az általam megvalósított módszer elmélete található. Az ötödik fejezetben a módszer megvalósítását mutatom be. A hatodik fejezetben tesztelésről, és az eredmények összehasonlításáról lesz szó. A dolgozat végén található az összefoglalás a munkámról, a hivatkozások, valamint a függelék, melléklet rész.