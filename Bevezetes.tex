\chapter{Bevezetés}
Az ütemezés feladatával az élet számos területén találkozunk, kezdve az egyszerű, hétköznapi problémáktól, mint például egy napon elvégzendő feladataink sorrendjének beosztása, professzionális sportcsapatok heti edzésprogramjának kialakításán át, egészen az ipari üzemek működéséig, ahol a rendelkezésre álló berendezésekhez kell rendelni az előállítani kívánt termékeket.
Bár az életünk különböző területein fellépő ütemezési problémák különböznek egymástól, bizonyos mértékben hasonlóság is fellelhető közöttük.
Eltérés lehet az ütemezési feladatok célfüggvénye.
Továbbá az adott ütemezési probléma lehet online, offline, illetve sztochasztikus vagy determinisztikus.
Minden fellépő probléma esetében az a cél, hogy az elvégzendő feladatokat a rendelkezésre álló erőforrások között megosszuk oly módon, hogy adott intervallumon belül a lehető legjobb megoldást kapjuk.
Fontos az, hogy ezt úgy tegyük meg, hogy a folyamat során fellépő korlátokat betartjuk, azokat nem sértjük meg.
Az ipari ütemezés két leggyakoribb célja a makespan minimalizálás, és a throughput maximalizálás.
A szakirodalomban legtöbb esetben a \textit{makespan} kifejezést az idő minimalizálására, a \textit{throughput} kifejezést pedig a termelés során előállított mennyiség maximalizálására alkalmazzák.

Az ipari gyártási folyamatok ütemezésére különböző módszerek léteznek már.
Ezek közé tartoznak a MILP (Mixed Integer Linear Programming - vegyes egészértékű lineáris programozás) megoldó módszerek, amelyek a lineáris programozáson alapulnak.
Továbbá ide sorolhatóak az időzített automaták, Petri hálók, valamint az S-gráf megoldó módszer, amely a munkám során a legfontosabb szerepet tölti be a felsorolt megoldó módszerek közül.
Dolgozatom második fejezetében ezek a módszerek kerülnek bemutatásra.
A harmadik részben a probléma definiálása történik meg, a negyedikben pedig az általam megvalósított módszer elmélete található.
Az ötödik fejezetben a módszer megvalósítását és szoftverbe való beillesztését mutatom be.
A hatodik fejezetben tesztelésről, és az eredmények összehasonlításáról lesz szó.
A dolgozat végén található a munkám összefoglalása, a hivatkozások, valamint a függelék.