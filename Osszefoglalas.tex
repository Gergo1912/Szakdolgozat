\chapter{Összefoglalás}
A dolgozatomban a throughput maximalizálás szakaszos üzemű rendszerekben témakörrel foglalkoztam. Először az ütemezéssel kapcsolatos irodalmat tanulmányoztam, amelynek során megismerkedtem az ipari környezetben használt ütemezési megoldó módszerekkel. Ezek közül az S-gráf keretrendszer lett az, amely a dolgozatom alapját képezi. Ezt követően megismerkedtem a a makespan minimalizálás és a throughput maximalizálás algoritmusával. Ezek megismerése adta meg az alapot, hogy elkészülhessen a flexibilis batchmérettel rendelkező feladatok megoldására alkalmas módszertan. Ez egy már meglévő S-gráf keretrendszert megvalósító szoftverbe történő implementálással valósult meg. Az implementáció megvalósítását követően tesztelésnek vetettem alá a bővített rendszert. A tesztelés eredményei bizonyították, hogy megfelelően működik a megoldó módszer, az jó eredményeket szolgáltat.

A feladat elvégzését követően a megoldó szoftver már képes flexibilis batchmérettel rendelkező feladatok megoldására, azonban további lehetőségek vannak, hogy a módszer gyorsabban elvégezhesse a feladatát, megtalálja az optimális megoldást a problémára. Ha az eddigi meglévő algoritmus módosításra kerülne úgy, hogy ha az adott rész nem ad jobb megoldást a meglévőnél, de mégis megvizsgáljuk, hogy megoldhatóak az ott lévő batchek akkor változhat és jobb teljesítmény lenne elérhető a módszerrel.