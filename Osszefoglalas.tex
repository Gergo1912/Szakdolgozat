\chapter{Összefoglalás}
A dolgozatomban a throughput maximalizálás szakaszos üzemű rendszerekben témakörrel foglalkoztam.
Először az ütemezéssel kapcsolatos irodalmat tanulmányoztam, amelynek során megismerkedtem az ipari környezetben használt ütemezési megoldó módszerekkel.
Ezek közül az S-gráf keretrendszer lett az, amely a dolgozatom alapját képezi.
Ezt követően megismerkedtem a makespan minimalizálás és a throughput maximalizálás algoritmusával.
Ezek megismerése adta meg az alapot, hogy elkészülhessen a flexibilis batch mérettel rendelkező feladatok megoldására alkalmas módszer.
Ez egy már meglévő S-gráf keretrendszert megvalósító szoftverbe történő implementálással valósult meg.
Az implementáció megvalósítását követően tesztelésnek vetettem alá a bővített rendszert.
A tesztelés eredményei bizonyították, hogy megfelelően működik a megoldó módszer, mivel jó eredményeket szolgáltat.
Ennek az új módszernek a segítségével idő spórolható meg az előfeldolgozó lépések során, mert már nem szükséges a diszkretizációs folyamat elvégzése.
Azonban a futási sebessége nem hozott jobb eredményeket, mint a régi megoldó módszer.
A jövőbeni tervek közé tartozik ennek az időnek a csökkentése.
Ehhez olyan keresési módszereket kell kifejleszteni, amelyekkel csökkenthető a keresési térben lévő konfigurációk száma, illetve egy adott konfiguráción belüli számítások csökkentésére alkalmasak.